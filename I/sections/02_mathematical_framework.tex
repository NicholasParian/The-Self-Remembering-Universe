\section{Mathematical Framework}
\label{sec:mathematical-framework}

We formalize the evolution of the universe as a recursive quantum system governed by a non-Markovian transition kernel \( K(\phi, \phi') \), entropic constraints, and a recursive action principle. The configuration space is composed of field, geometric, and informational degrees of freedom, with memory effects implemented through attractor-constrained recursion.

\subsection{Recursive Transition Amplitudes}

The core update equation for the recursive state is:
\begin{equation}
\Psi_n(\phi) = \int K(\phi, \phi') \Psi_{n-1}(\phi') \, d\phi'
\end{equation}
The transition kernel \( K(\phi, \phi') \) is derived from the large-spin asymptotics of the EPRL spinfoam model and encodes interference-filtered evolution across Einstein–Rosen bridges:
\begin{equation}
K(\phi, \phi') \sim \exp\left[i S_{\text{ERB}}(\phi, \phi')\right] \cdot \mathcal{F}(\phi, \phi')
\end{equation}
Here:
\begin{itemize}
    \item \( S_{\text{ERB}}(\phi, \phi') \) is the entropic bridge action, modeling area and curvature matching across cycles (Appendix~C.3),
    \item \( \mathcal{F}(\phi, \phi') \) is a Gaussian coherence filter modulating allowed transitions (Appendix~C.4).
\end{itemize}
The mapping from cosmological variables to LQG boundary data is detailed in Appendix~C.2: scale factor \( a \mapsto \sqrt{j(j+1)} \), scalar field \( \varphi \) to vertex scalar data, coherence \( \lambda \) to fidelity weights, and entanglement \( E \) to ERB throat area.

\subsection{Configuration Space and Geometry}

The configuration vector is defined as:
\begin{align}
\phi &= \{a, \varphi, \lambda, E\} \\
\lambda &:= \left| \left\langle \Psi_n \middle| \Psi_{n-1} \right\rangle \right|^2 \\
E &:= \sqrt{S(\rho_{\text{red}})} = \sqrt{-\mathrm{Tr}(\rho_{\text{red}} \log \rho_{\text{red}})}
\end{align}
where \( \lambda \) quantifies attractor fidelity and encodes phase-coherence persistence across cycles. Unless otherwise stated, we assume a flat field-space metric \( G_{ab} = \delta_{ab} \) consistent with the semiclassical LQC regime.

\subsection{Decoherence Kernel Dynamics}

The intra-cycle memory kernel is defined as:
\begin{equation}
D(\tau, E) = \gamma(E) \, e^{-\tau/\tau_c(E)} \cos(\omega_0(E) \tau)
\end{equation}
with entropy-dependent coefficients:
\begin{align*}
\tau_c(E) &\sim E^{-1}, &
\gamma(E) &\sim e^{-\beta / E}, &
\omega_0(E) &\sim \sqrt{1 - \left( \frac{E}{E_{\text{max}}} \right)}
\end{align*}
This enforces a finite coherence window with modulation tied to entanglement strength.

\subsection{Information Divergence Term}

We regulate entropy growth between non-aligned configurations using quantum relative entropy:
\begin{equation}
I(\phi, \phi') := S(\rho_\phi \| \rho_{\phi'}) = \mathrm{Tr}\left[ \rho_\phi \left( \log \rho_\phi - \log \rho_{\phi'} \right) \right]
\end{equation}
This replaces heuristic overlap terms (e.g., \( \log |\langle \Psi_n | \Psi_{n-1} \rangle| \)) with a formal divergence metric.

\subsection{Recursive Entropy and Memory Conservation}

The total entropy per cycle is modeled as:
\begin{equation}
S_n = \frac{A_{n-1}}{4G\hbar} + \lambda_S S(\rho_n \| \rho_{n-1})
\end{equation}
and is constrained by:
\begin{equation}
S_{n+1} \leq S_n - S_{\text{BH}} + \Delta S_{\text{holo}}
\end{equation}
During high-information-gain phases, the recursive string tension \( \lambda_n \) increases. Its derivative acts as a thermodynamic regulator:
\begin{equation}
\frac{dS_n}{dn} \sim -\frac{d\lambda_n}{dn}
\end{equation}
Memory stress is balanced by radiative entropy emission:
\begin{equation}
\frac{dS_{\text{net}}}{dn} = \frac{dS_{\text{rad}}}{dn} - \frac{d\lambda_n}{dn} \approx 0
\end{equation}
This constraint defines a stable attractor corridor in thermodynamic phase space.

\subsection{Energy Transfer Across Cycles}

Energy flux through the bridge is given by:
\begin{equation}
\Delta E_n = \frac{\kappa \Delta A}{8\pi G} + T_H \Delta S_{\text{holo}} - \lambda_E I(\phi, \phi') + \kappa_\lambda \lambda_n^2
\end{equation}
Here, the final term models coherence tension release (e.g., via supernova-class rupture events).

\subsection{Attractor Convergence Criteria}

The attractor satisfies:
\begin{equation}
\Psi^*(\phi) = \int K(\phi, \phi') \Psi^*(\phi') \, d\phi'
\end{equation}
Its convergence is governed by the fitness functional:
\begin{equation}
\mathcal{F}_n = \alpha_C \, \text{Tr}(\rho_n^2) - \alpha_S S_n + \alpha_M \lambda
\end{equation}
with conditions:
\begin{equation}
\frac{d\mathcal{F}_n}{dn} > 0, \quad \mathcal{F}_n \geq \mathcal{F}_{\text{crit}}
\end{equation}

\subsection{Observational Predictions}

This framework yields:
\begin{itemize}
    \item \textbf{CMB suppression at low multipoles:} due to Gaussian filtering in \( K(\phi, \phi') \),
    \item \textbf{Gravitational wave echo patterns:} with frequencies \( f_j \sim \sqrt{j(j+1)} / (2\pi \ell_P) \),
    \item \textbf{Non-Gaussianity:} scale-dependent \( f_{NL}^{\text{rec}} \sim \lambda_E \lambda^\alpha \),
    \item \textbf{EB-mode polarization:} void-aligned from coherence echoes.
\end{itemize}
The attractor \( \Psi^*(\phi) \) also serves as the dynamical backbone for field unification in Paper II.

\subsection{Limitations and Extensions}

This formalism currently assumes:
\begin{itemize}
    \item Large-spin approximations for spinfoam amplitudes,
    \item Gaussian coherence filters,
    \item Semiclassical thermodynamics.
\end{itemize}
Appendix~C details future improvements using full group field theory amplitudes and numerical convergence simulations.

\subsection{Tension-Induced Collapse Criterion}

We define the falsifiable rupture threshold:
\begin{equation}
\lambda_n > \lambda_{\text{crit}} \quad \Rightarrow \quad \text{burst at } f_{\text{burst}} \sim \frac{1}{\tau_{\text{mem}}}
\end{equation}
Detectable high-frequency GW bursts linked to void collapse or coherence decay would empirically confirm this constraint.

