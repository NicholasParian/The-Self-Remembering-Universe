\section{Forecasting and Simulation Framework}
\label{sec:forecasting}

This section outlines a methodology for simulating recursive quantum cosmology and generating observational forecasts. It connects the core kernel–attractor formalism to empirical targets across the cosmic microwave background (CMB), gravitational wave (GW) spectrum, and large-scale structure (LSS). The goal is to enable falsifiable predictions grounded in the dynamics of recursive memory and entropy regulation.

\subsection{Kernel Parameter Mapping to Observables}

Each component of the recursive transition kernel \( K(\phi, \phi') \) contributes to distinct cosmological observables. The table below defines explicit parameter-to-observable mappings:

\begin{table}[H]
\centering
\begin{tabular}{lll}
\toprule
\textbf{Kernel Parameter} & \textbf{Model Quantity} & \textbf{Observable Signature} \\
\midrule
\( \sigma_\varphi \) & Scalar field coherence width & CMB non-Gaussianity \( f_{\text{NL}} \) \\
\( \sigma_E \) & Entanglement filtering width & EB-mode polarization alignment \\
\( j_0 \) & Dominant spinfoam spin scale & GW spectral suppression dips at \( f_j \sim \sqrt{j(j+1)} / 2\pi \ell_{\text{Pl}} \) \\
\( \lambda_n \) & Cycle-to-cycle coherence fidelity & CMB low-\( \ell \) power suppression \\
\( \tau_M \sim E \) & Coherence delay scale & Delayed GW bursts from memory collapse \\
\bottomrule
\end{tabular}
\caption{Mapping of kernel parameters to cosmological observables.}
\end{table}

\subsection{Numerical Simulation Strategy}

Numerical evolution of \( \Psi_n(\phi) \) and its convergence to the attractor \( \Psi^*(\phi) \) follows the architecture described in Appendix~\ref{appendix:C5}. Core simulation steps include:

\begin{itemize}
    \item \textbf{Crank–Nicolson integration} of the LQC minisuperspace Hamiltonian over \( (a_n, \varphi_n) \).
    \item \textbf{Spin-sum Monte Carlo sampling} over dominant spin sectors \( j \sim j_0 \), filtered by coherence constraints.
    \item \textbf{Finite-memory convolution} for the non-Markovian kernel \( D(\tau, E) \), updated dynamically from entropy profiles.
    \item \textbf{Attractor diagnostics:}
    \[
    \mathcal{L}_n := \log \left( \frac{\|\Psi_{n+1} - \Psi_n\|}{\|\Psi_n - \Psi_{n-1}\|} \right), \quad \lambda_n := |\langle \Psi_n | \Psi_{n-1} \rangle|^2
    \]
    \item \textbf{Entropy- and tension-constrained phase-space sampling}, enforcing physical admissibility.
\end{itemize}

\subsection{Likelihood Function Templates}

To connect the kernel model to empirical constraints, we define likelihood functions:

\paragraph{(1) CMB Low-\( \ell \) Power Suppression:}
\[
\mathcal{L}_{\text{CMB}}(\theta) = \prod_{\ell < 30} \frac{1}{\sqrt{2\pi \sigma_\ell^2}} \exp\left[ -\frac{(C_\ell^{\text{obs}} - C_\ell^{\text{rec}}(\theta))^2}{2\sigma_\ell^2} \right]
\]

\paragraph{(2) Gravitational Wave Spectrum Suppression:}
\[
\mathcal{L}_{\text{GW}}(j_0) = \prod_j \exp\left[ -\frac{(\Omega_{\text{GW}}^{\text{obs}}(f_j) - \Omega_{\text{GW}}^{\text{rec}}(f_j))^2}{2\sigma_j^2} \right]
\]

\paragraph{(3) EB-Mode Polarization Alignment:}
\[
\mathcal{L}_{\text{EB}}(\sigma_E) = \prod_\ell \exp\left[ -\frac{(C_\ell^{EB,\text{obs}} - C_\ell^{EB,\text{rec}})^2}{2\sigma_\ell^2} \right]
\]

These enable parameter inference via MCMC or nested sampling, compatible with \texttt{Cobaya}, \texttt{PolyChord}, or \texttt{emcee}.

\subsection{Forecast Prioritization}

The most predictive and falsifiable targets include:

\begin{itemize}
    \item \textbf{Low-\( \ell \) CMB suppression:} Memory-filtered damping from reduced \( \lambda_n \), measurable by Planck and LiteBIRD.
    \item \textbf{GW spectrum nulls:} Interference dips at \( f_j \sim \sqrt{j(j+1)} / 2\pi \ell_{\text{Pl}} \), testable by LISA and DECIGO.
    \item \textbf{Recursive non-Gaussianity:} Coherence-modulated \( f_{\text{NL}} \) scaling, detectable via CMB-S4.
    \item \textbf{Void-aligned EB polarization:} Entanglement-induced asymmetries in void regions, testable via Euclid and SKA lensing.
    \item \textbf{Delayed GW bursts:} Tension-induced coherence collapse signals at \( f \sim 1/\tau_M \), observable in pulsar timing arrays and LISA.
\end{itemize}

\subsection{Codebase and Simulation Toolkit}

A simulation platform will be released containing:

\begin{itemize}
    \item Recursive kernel and attractor solver (Python and Julia),
    \item Spinfoam-based GW feature generator,
    \item CMB module integrating with \texttt{CAMB} or \texttt{CLASS},
    \item Full inference interface for likelihood comparison and forecast validation.
\end{itemize}

This toolkit enables direct comparison between recursive cosmology predictions and datasets from \textbf{LiteBIRD}, \textbf{CMB-S4}, \textbf{LISA}, and \textbf{Euclid}, supporting rigorous testing of the model’s empirical viability.
