\section{Introduction}
\label{sec:Intro}

We propose a recursive cosmological model in which the universe evolves through coherence-preserving transitions between quantum geometries. This framework integrates three foundational pillars: (1) loop quantum cosmology (LQC)~\cite{ashtekar2006quantum}, which provides a discrete geometric bounce mechanism; (2) Einstein–Rosen bridge (ERB) thermodynamics~\cite{maldacena2013cool}, which mediates entanglement-based memory transfer across cycles; and (3) non-Markovian decoherence~\cite{breuer2002theory}, which regulates entropy production through memory-sensitive filtering.

The state of the universe at each cycle \( n \) is represented by a recursive wavefunction \( \Psi_n(\phi) \), where:
\[
\phi = (a, \varphi, \lambda, E)
\]
encodes the discrete scale factor \( a \), scalar field configuration \( \varphi \), memory fidelity \( \lambda \), and entanglement eigenvalue \( E \) across the ER bridge.

Recursive evolution is governed by a transition kernel \( K(\phi, \phi') \), formally derived from spinfoam amplitudes. The effective kernel includes entropy divergence penalties, coherence overlap weighting, and a Gaussian filtering envelope over configuration space. The corresponding normalized operator defines a unique fixed-point attractor \( \Psi^*(\phi) \), whose existence and convergence are proven via contraction mapping.

This attractor satisfies a variational principle that balances entropy production, recursive tension, and memory fidelity. A thermodynamic compensation constraint is imposed:
\[
\Delta S_{\text{gain}} + \Delta S_{\text{rad}} = \Delta S_{\text{exp}},
\]
which ensures that information gain from coherence sharpening is offset by radiative emission and expansion-induced entropy dilution. Violation of this constraint triggers recursive collapse—interpreted as supernova-class coherence rupture or black hole formation.

Observable predictions of the model arise directly from kernel and attractor structure. These include:
\begin{enumerate}
    \item Suppression of the CMB power spectrum at low \( \ell \),
    \item Quantized dips in the stochastic gravitational wave background at spin-correlated frequencies,
    \item Recursive non-Gaussianity in the CMB bispectrum, modulated by memory fidelity,
    \item Void-aligned EB-mode CMB polarization induced by entanglement structure.
\end{enumerate}
These predictions offer falsifiable discrimination from inflationary and conformal cyclic scenarios.

In sum, the Self-Remembering Universe formalism reframes cosmological evolution as a memory-driven quantum process. It provides a mathematically complete, observationally testable framework integrating quantum gravity, thermodynamics, and recursive information theory.
