\section{Introduction}
\label{sec:Intro}

We propose a recursive cosmological model in which the universe evolves through coherence-preserving transitions between quantum geometries. This framework integrates three foundational pillars: (1) loop quantum cosmology (LQC)~\cite{ashtekar2006quantum}, which provides a discrete geometric bounce mechanism; (2) Einstein–Rosen bridge (ERB) thermodynamics~\cite{maldacena2013cool}, which mediates entanglement-based memory transfer across cycles; and (3) non-Markovian decoherence~\cite{breuer2002theory}, which regulates entropy production through memory-sensitive filtering.

The state of the universe at each cycle \( n \) is represented by a wavefunction \( \Psi_n(\phi) \), where:
\[
\phi = (a, \varphi, \lambda, E)
\]
encodes the discrete scale factor \( a \), scalar field \( \varphi \), memory fidelity \( \lambda \), and entanglement eigenvalue \( E \) across the bridge.

Recursive evolution is governed by a transition kernel \( K(\phi, \phi') \) formally derived from spinfoam amplitudes. The effective kernel includes exponential penalties for entropy divergence and rewards for coherence overlap, filtered by a Gaussian envelope in field-space and curvature alignment. The normalized recursive operator defines a unique fixed-point attractor \( \Psi^*(\phi) \), whose existence and convergence are proven via contraction mapping.

This attractor satisfies a variational principle balancing entropy production, coherence tension, and fidelity to prior cycles. We impose a thermodynamic constraint—the entropy–tension compensation identity—which enforces that entropy gain due to memory sharpening must be offset by radiative emission and geometric dilution. Violation of this balance triggers recursive instability, modeled as supernova-like coherence rupture or black hole collapse events.

Observable predictions of this model arise directly from the structure of the kernel and attractor filtering. They include: (1) suppression of the CMB power spectrum at low \( \ell \), (2) quantized dips in the gravitational wave background at spin-correlated frequencies, (3) recursive non-Gaussianity modulated by memory fidelity, and (4) void-aligned EB-mode CMB polarization. These signatures offer falsifiable discrimination from both inflationary and conformal cyclic scenarios.

In sum, the Self-Remembering Universe formalism reframes cosmological evolution as a memory-driven quantum process. It offers a mathematically complete, observationally testable framework for integrating quantum gravity, thermodynamics, and cosmological recursion.
