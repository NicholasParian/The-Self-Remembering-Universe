\section{Interpretation, Limitations, and Future Directions}
\label{sec:interpretation}

\subsection{Interpretation of the Framework}

This framework models cosmological evolution as a recursive process constrained by memory-preserving transition kernels, entropic filtering, and coherence tension bounds. Each cycle encodes partial information from prior epochs via quantum overlaps \( \lambda_n = |\langle \Psi_{n-1} | \Psi_n \rangle|^2 \), with memory fidelity constrained by entanglement structure and entropy divergence.

Time is emergent and relational, arising from coherence propagation across cycles. The fixed-point attractor \( \Psi^*(\phi) \), derived in Appendix~\ref{appendix:C}, represents the asymptotic limit of recursive evolution under entropy-regulated interference.

The recursive action principle enforces thermodynamic consistency via:
\[
\Delta S_{\text{fwd}} = \Delta S_{\text{mem}},
\]
where forward entropy increase is compensated by memory retention constraints. Tension \( \lambda_n \) increases with information gain, while cosmological expansion induces entropy dilution. Collapse occurs when \( \lambda_n > \lambda_{\text{crit}} \), corresponding to dimensional string rupture or memory failure. Hawking radiation regulates this feedback, maintaining coherence viability across bounces.

\subsection{Model Limitations}

While internally consistent, the model retains several areas for further development:

\begin{itemize}
  \item The kernel \( K(\phi, \phi') \) is derived in the large-spin approximation of the EPRL model, without quantum group or topology-change corrections.
  \item Entanglement fidelity \( \lambda_n \) is computed via state overlap; a full entanglement Hamiltonian has not yet been derived.
  \item Observer dynamics are encoded via projection operators \( \hat{O}_n \) and entanglement tensors \( O_n \), but not derived from a complete open-system formalism.
  \item The 12-dimensional embedding (Appendix~\ref{appendix:A}) provides a consistent geometric structure, but lacks independent derivation from string/M-theory compactification pathways.
  \item The Euler–Lagrange equations governing recursive tension and entropy gradients are partially specified and require full dynamical derivation.
\end{itemize}

These limitations suggest focused directions for formalization, but do not compromise the coherence or falsifiability of the core framework.

\subsection{Comparison to Existing Cosmological Models}

Relative to other paradigms:

\begin{itemize}
  \item Inflationary models externalize initial conditions and require fine-tuned potentials; recursive memory evolution internalizes this structure via \( \Psi_{n-1} \to \Psi_n \).
  \item Loop quantum cosmology provides the bounce mechanism, but lacks an entropy-coherence feedback loop. Our model extends LQC by embedding memory propagation within the kernel structure.
  \item Conformal cyclic cosmology proposes inter-cycle continuity via conformal rescaling. In contrast, this model enforces memory conservation via entanglement fidelity and quantifies recursive collapse via thermodynamic tension.
\end{itemize}

The recursive model provides a falsifiable alternative grounded in quantum gravity and information theory, with testable predictions spanning gravitational wave, CMB, and large-scale structure observables.

\subsection{Role of the Observer}

Observers are modeled not as classical agents but as embedded quantum subsystems. The observer projection operator \( \hat{O}_n \) defines entangled measurement subspaces on boundary slices, while the entanglement tensor \( O_n \) modulates the transition kernel:
\[
K(\phi, \phi') \to K_{O_n}(\phi, \phi') = K(\phi, \phi') \cdot \langle O_n(\phi) | O_{n-1}(\phi') \rangle.
\]

This structure encodes decoherence history and recursive selection without invoking anthropic or external assumptions. Extensions may involve quantum reference frames, subsystem encodings, and path-integral formulations with informational constraints.

\subsection{Future Research Directions}

Priority research targets include:

\begin{itemize}
  \item Derivation of full recursive Euler–Lagrange equations from the constrained action \( \delta \mathcal{A}_{\text{total}} = 0 \),
  \item Mapping the basin of convergence for the attractor \( \Psi^*(\phi) \) under entropy variation and coherence decay,
  \item Operational characterization of entanglement eigenvalue \( E \) and memory fidelity \( \lambda_n \) in terms of measurable observables,
  \item Integration of the entropy-modulated memory kernel \( D(\tau, E) \) into late-time radiative dynamics,
  \item Extension of the spinfoam kernel to incorporate quantum group deformations, topology-change contributions, and full compactification dynamics from M-theory embeddings.
\end{itemize}

These steps will refine the theoretical foundation, clarify observational implications, and position the recursive cosmology framework within the broader landscape of testable quantum gravitational theories.
