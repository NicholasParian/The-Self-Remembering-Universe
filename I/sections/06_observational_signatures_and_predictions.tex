
\section{Observational Signatures and Predictions}
\label{sec:observables}

The recursive cosmology framework yields falsifiable observational predictions across multiple cosmological channels. These arise from non-Markovian memory propagation, entropy-regulated evolution, and phase interference encoded in the transition kernel \( K(\phi, \phi') \). The kernel structure is derived from spinfoam amplitudes with embedded configuration variables \( \phi = (a, \varphi, \lambda, E) \), as detailed in Appendix~\ref{appendix:C}.

Each signature emerges from a distinct parameter of the transition kernel, coherence fidelity, or recursive entropy relation, and is traceable to physical constructs defined in Sections~\ref{sec:kernel}–\ref{sec:recursive-action-formal}.

\subsection{6.1 Coherence-Driven CMB Suppression}

The Gaussian filtering structure of \( K(\phi, \phi') \) introduces damping of long-wavelength curvature perturbations. Specifically:
\[
\mathcal{F}(\phi, \phi') = \exp\left[-\frac{(a - a')^2}{2\sigma_a^2} - \frac{(\varphi - \varphi')^2}{2\sigma_\varphi^2} - \frac{(E - E')^2}{2\sigma_E^2}\right]
\]
suppresses correlations in the Sachs–Wolfe regime, corresponding to the large-angle anisotropies in the CMB. This leads to a characteristic suppression of the angular power spectrum for multipoles \( \ell < 30 \), consistent with Planck observations~\cite{planck2019inflation}.

The amplitude of suppression scales with memory fidelity \( \lambda_n = |\langle \Psi_{n-1} | \Psi_n \rangle|^2 \). The effective form is:
\[
C_\ell^{\text{rec}} \approx C_\ell^{\text{std}} \cdot \lambda_n \cdot \exp\left(-\frac{\ell^2}{2\sigma_\ell^2}\right)
\]
where \( \sigma_\ell \) is the harmonic-space coherence width.

\subsection{6.2 Gravitational Wave Interference Nulls}

The kernel’s discrete spin structure induces frequency-selective suppression in the stochastic gravitational wave background (SGWB). Suppression frequencies correspond to dominant spin contributions:
\[
f_j \sim \frac{\sqrt{j(j+1)}}{2\pi \ell_{\text{Pl}}}, \quad j \in \mathbb{N}
\]
These arise from interference cancellation at specific spin modes in the kernel amplitude sum. They yield testable nulls or dips in \( \Omega_{\text{GW}}(f) \), especially in the LISA band \( f \sim 10^{-3} \,\text{Hz} \), with suppression depths \( \Delta \Omega_{\text{GW}} \sim 10^{-12} \)~\cite{amaroseoane2017laser}.

\subsection{6.3 Recursive Non-Gaussianity in the CMB}

Recursive phase coherence modulates the bispectrum, leading to scale-dependent local-type non-Gaussianity:
\[
f_{\text{NL}}^{\text{rec}}(k) = \lambda_E \left|\langle \Psi_{n-1} | \Psi_n \rangle\right|^\alpha
\]
where \( \alpha \) encodes the steepness of coherence sensitivity, and \( \lambda_E \) is the entanglement coupling (see Section~\ref{sec:recursive-action-formal}). This yields observable \( f_{\text{NL}} \sim 0.5 - 10 \) for sufficiently coherent transitions, within detection range of CMB-S4~\cite{cmbs4forecast2019} and LiteBIRD~\cite{litebird2023}.

\subsection{6.4 EB-Mode Polarization from Entangled Void Structure}

Phase-coherent configurations seeded during recursive cycles generate aligned voids and parity-violating CMB polarization:
\[
\langle C_\ell^{EB} \rangle \propto \lambda_E^2 \left|\langle \Psi_{n-1} | \Psi_n \rangle\right|^2
\]
Such EB-mode alignment cannot be generated by scalar inflationary perturbations, making it a clean discriminator of recursive entanglement structure. This effect is absent in inflation and CCC and can be tested via stacked void lensing analysis in SKA and Euclid surveys~\cite{dewdney2009ska, laureijs2011euclid}.

\subsection{6.5 Entropy Retention and Memory Saturation}

The recursive entropy relation:
\[
S_{n+1} = \frac{A_{n}}{4G\hbar} + \lambda_S \, S(\rho_{n+1} \| \rho_n)
\]
predicts bounded entropy growth across cycles. Memory propagation across bounces is mediated by ERB entanglement and coherence-fidelity weighting, as formalized in the recursive action. The entropy bound condition implies:
\[
S_{n+1} \leq S_n + \Delta S_{\text{Hawking}} - \Delta S_{\text{decoherence}}
\]
This constraint suppresses disorder propagation and ensures attractor convergence. Late-time void alignments and low-entropy cold spots in the CMB may reflect this constraint~\cite{almheiri2019entropy}.

\subsection{6.6 Falsifiability and Measurement Targets}

Key null tests include:
\begin{itemize}
    \item Absence of CMB power suppression at \( \ell < 30 \)
    \item Lack of scale-dependent \( f_{\text{NL}} \) in bispectrum
    \item No EB-mode polarization in large-scale voids
    \item No dips in SGWB spectrum at \( f_j \sim \text{Planck-derived} \)
    \item No impulsive gravitational wave bursts at frequencies \( f \sim 1/\tau_{\text{mem}} \) corresponding to tension-induced coherence collapse (\( \lambda_n > \lambda_{\text{crit}} \))
\end{itemize}

\subsection{6.7 Summary}

The recursive model predicts:
\begin{itemize}
    \item Suppressed CMB power at low-\( \ell \) via field coherence damping
    \item Discrete suppression frequencies in the SGWB
    \item Phase-driven non-Gaussianity in CMB bispectrum
    \item Parity-violating EB polarization from entangled voids
    \item Bounded entropy growth across cosmological cycles
    \item Supernova-class gravitational wave bursts triggered by coherence tension collapse, with frequencies set by the memory kernel and amplitudes linked to entropy loss
\end{itemize}

These features define a falsifiable observational signature space distinct from both inflation and ekpyrotic alternatives. The model’s predictions will be tested by upcoming CMB (LiteBIRD, CMB-S4), GW (LISA), and large-scale structure (SKA, Euclid) experiments.
