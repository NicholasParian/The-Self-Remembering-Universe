\section{Recursive Observation and Entanglement Symmetry}
\label{sec:recursive-observation}

\subsection{7.1 Observation as a Recursive Boundary Condition}

Observation in this model is endogenous: each cycle acts as a boundary condition for the next. The recursive kernel \( K(\phi, \phi') \) governs the transition from state \( \Psi_{n-1} \) to \( \Psi_n \), and this transition is modulated by overlap fidelity:
\[
\lambda_n = |\langle \Psi_{n-1} | \Psi_n \rangle|^2
\]
Rather than modeling trajectory collapse, this framework defines a self-referential cosmology, consistent with quantum Darwinism~\cite{zurek_environment-induced_2003,zurek_quantum_2009}. Observation is expressed as recursive filtering of field configurations across cycles via coherence.

\subsection{7.2 Entanglement as a Temporal Constraint}

The entanglement variable \( E \) (Section~\ref{sec:recursive-action-formal}) modulates decoherence timing, entropy production, and recursive stability. High \( E \) implies long coherence times, slow entropy growth, and closer approach to the attractor \( \Psi^*(\phi) \). The memory kernel \( D(\tau, E) \) is governed by this entanglement eigenvalue, with delay time \( \tau_c \sim E \). Thus, entanglement functions as a regulator of temporal geometry.

Motion through high-curvature regions of the scalar field landscape (e.g., near Higgs-like couplings) compresses internal coherence, leading to accelerated decoherence. This provides a geometric explanation for time dilation as a coherence-dependent deformation of temporal structure.

\subsection{7.3 Black Holes, Collapse, and Recursive Reinitialization}

Recursive coherence failure occurs when:
\[
\lambda_n \to 0, \quad S_n \to S_{\text{max}}, \quad R \to R_{\text{crit}}
\]
These conditions define a collapse surface in configuration space. Black holes correspond to entropy sinks where fidelity vanishes and recursive propagation terminates. The ERB fails to transmit structure, and the system either projects to a null state or reinitializes with zero coherence:
\begin{itemize}
    \item transition to a decohered fixed point (null configuration), or
    \item projection into a new initial state with \( \lambda_0 = 0 \)
\end{itemize}
This collapse corresponds to a loss of constructive overlap in the kernel:
\[
K(\phi, \phi') \to \delta(\phi - \phi')
\]

\subsection{7.4 Recursive Coherence Darwinism}

We define the principle governing recursive selection as:
\begin{quote}
\textbf{Recursive Coherence Darwinism:} Only field configurations that preserve coherence across cycles are propagated forward; decohered branches are statistically suppressed by entropy penalties in the recursive action.
\end{quote}

Mathematically, survival probability is encoded via the fitness functional \( \mathcal{F}_n \) (Appendix~\ref{appendix:C}):
\[
P_{\text{survive}}(F_n) = \frac{1}{1 + e^{\kappa(F_{\text{crit}} - F_n)}}
\]
This expression emerges naturally from the variation of the recursive entropy penalty term, acting as a soft coherence threshold in the configuration space path integral. Attractor convergence occurs when this probability stabilizes, and the system self-selects configurations with high fidelity, low entropy, and sustained entanglement.

\subsection{7.5 Observer Tensor and Kernel Modulation}

The observer tensor \( O_n \) is a map from configuration space to entangled subsystem partitions, tracking environment-induced structure. It modulates kernel evolution as:
\[
K(\phi, \phi') \mapsto K_{O_n}(\phi, \phi') = K(\phi, \phi') \cdot \langle O_n(\phi) | O_{n-1}(\phi') \rangle
\]
This formulation allows for recursive encoding of decoherence structure, analogous to environment-induced superselection. Observers are not classical agents but embedded subsystems whose entanglement history conditions transition amplitudes.

This tensor evolves under a decoherence-modulated rule:
\[
O_{n+1} = \mathcal{U}_n(O_n) + \delta O_n
\]
where \( \mathcal{U}_n \) is a unitary channel conditioned on the memory kernel and entanglement fidelity, and \( \delta O_n \) encodes entropy-induced noise. This structure ensures that observer influence on recursion is consistent with the underlying informational dynamics.

\subsection{7.6 Summary}

Recursive observation arises not from classical measurement, but from entangled continuity between cosmological cycles. Key mechanisms include:

\begin{itemize}
    \item Boundary-induced overlap selection via \( K(\phi, \phi') \)
    \item Temporal regulation through entanglement-dependent decoherence
    \item Collapse and reinitialization at coherence thresholds
    \item Selection pressure on coherent cycles via fitness functional \( \mathcal{F}_n \)
    \item Observer effects encoded in evolving entanglement tensors \( O_n \)
\end{itemize}

This section bridges quantum cosmology and information theory, showing how recursive memory structure—not external measurement—governs the persistence and propagation of cosmological configurations.
