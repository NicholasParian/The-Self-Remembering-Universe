\section{Theoretical Comparisons and Open Problems}
\label{sec:comparisons}

\subsection{Relation to Other Frameworks}

This framework integrates and extends several foundational cosmological paradigms:
\begin{itemize}
    \item \textbf{Loop Quantum Cosmology (LQC)} provides the underlying bounce mechanism via quantized geometry~\cite{ashtekar2006quantum, bojowald2001absence}.
    \item \textbf{ER=EPR} connects entanglement structure to spacetime bridges, providing the geometric infrastructure for recursive memory transfer via Einstein–Rosen bridges~\cite{maldacena2013cool}.
    \item \textbf{Conformal Cyclic Cosmology (CCC)} inspires the idea of inter-cycle continuity, reformulated in terms of entanglement-preserving recursive dynamics~\cite{penrose2010cycles}.
    \item \textbf{Quantum Darwinism} motivates the selection principle driving long-term coherence and attractor formation~\cite{zurek_quantum_2009}.
\end{itemize}

Unlike prior models, this framework integrates quantum information dynamics directly into the variational structure of cosmological evolution. It is distinguished by its explicit modeling of memory propagation, entropy-constrained kernel structure, and emergence of a fixed-point attractor state.

\subsection{Comparison Table}

\begin{table}[H]
\centering
\begin{tabular}{|l|c|c|c|c|}
\hline
\textbf{Feature} & \textbf{Recursive Model} & \textbf{Inflation} & \textbf{CCC} & \textbf{LQC} \\
\hline
Bounce Mechanism & Quantized with memory filtering & Not applicable & Conformal reset & Quantized bounce \\
Entropy Control & Coherence-bound kernel & Monotonic growth & Conformal suppression & Area bounded \\
Gravitational Waves & Discrete interference dips & Smooth spectrum & Suppressed, featureless & Smooth spectrum \\
Non-Gaussianity & Phase-coherence modulated \( f_{\text{NL}} \) (via \( \lambda, \lambda_E \)) & Gaussian (low) & Conformal imprint & Weak suppression \\
Temporal Structure & Recursive attractor evolution & Forward-only & Conformal cycles & Time-symmetric \\
Observer Role & Embedded via entanglement tensor \( O_n \) & External or excluded & Postulated but not dynamically modeled & Implicit via quantum geometry \\
\hline
\end{tabular}
\caption{Comparison of theoretical structures across major cosmological paradigms}
\end{table}

\subsection{Open Theoretical Questions}

Key theoretical questions remain unresolved:
\begin{itemize}
    \item Can the full spinfoam kernel \( K(\phi, \phi') \) be derived from EPRL amplitudes with entanglement-labeled boundaries?
    \item How does the entropy constraint \( \Delta S_{\text{fwd}} = \Delta S_{\text{mem}} \) influence the renormalization of spin foam amplitudes?
    \item Under what conditions does the fixed-point attractor \( \Psi^*(\phi) \) become unique and globally stable?
    \item How is the entanglement eigenvalue \( E \) dynamically generated and physically measured?
    \item Can entropy-constrained path integrals across cycles be computed with asymptotic or saddle-point control?
    \item What is the geometric structure of the 12D memory boundary state \( |\Omega\rangle \)?
    \item What role do higher-order corrections in the large-spin expansion play in determining the stability of kernel-derived observables?
\end{itemize}

\subsection{Experimental Challenges}

Empirical challenges for the model include:
\begin{itemize}
    \item Achieving sufficient sensitivity to detect narrowband GW suppression at \( f_j \sim 10^{-3} \,\text{Hz} \),
    \item Disambiguating phase-dependent non-Gaussianity from scale-invariant models,
    \item Isolating void-induced EB correlations from lensing and systematics,
    \item Establishing cosmological constraints on entropy drift due to recursive decoherence,
    \item Validating the recursive entropy bound in observational CMB and LSS datasets under decoherence modeling assumptions,
    \item Determining whether apparent low-\( \ell \) anomalies in the CMB are statistically robust enough to differentiate from $\Lambda$CDM under this model.
\end{itemize}

\subsection{Future Directions}

Research priorities include:
\begin{itemize}
    \item Simulation of attractor convergence using LQC-sourced Hamiltonians and spin-weighted kernel integrals,
    \item Formal derivation of recursive Euler–Lagrange equations from \( \delta \mathcal{A}_{\text{total}} = 0 \),
    \item Construction of gauge-invariant entanglement observables linked to boundary conditions,
    \item Deeper analysis of the observer tensor \( O_n \) as a structural degree of freedom in decoherence and kernel evolution,
    \item Mapping the parameter space in which recursive attractor convergence occurs and identifying critical transitions between coherence-dominated and entropy-dominated regimes.
\end{itemize}

\subsection{Summary}

This framework builds on established quantum gravity principles to propose a novel structure for cosmological evolution: recursive memory filtering across bounce transitions, enforced by entropy constraints and realized through an attractor-driven coherence mechanism. Its viability depends on rigorous derivations and empirical confirmation of memory-dependent cosmological observables.
