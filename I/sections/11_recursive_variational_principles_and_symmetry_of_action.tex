\section{Recursive Variational Principles and Symmetry of Action}
\label{sec:recursive-variational-symmetry}

\subsection{Recursive Action Formalism}

Recursive evolution is governed by a total action principle that balances entropy accumulation and memory retention:
\begin{equation}
\delta \mathcal{A}_{\text{total}} = 0 \quad \text{subject to} \quad \Delta S_{\text{fwd}} = \Delta S_{\text{mem}}.
\end{equation}

The action per cycle is expressed using a recursive Lagrangian:
\begin{equation}
\mathcal{A}_n = \int dt \, \mathcal{L}_n(q_n, \dot{q}_n; q_{n-1}),
\end{equation}
with configuration variables \( q_n = (a_n, \varphi_n, \lambda_n, E_n) \) and Lagrangian contributions encoding geometry, matter fields, decoherence, and memory inheritance.

\subsection{Observer Projection and Boundary Constraint}

Observation is modeled as a projection condition at the boundary surface of decoherence:
\begin{equation}
\delta \Psi_n \big|_{\Sigma_{\text{obs}}} = \hat{O}_n \Psi_n,
\end{equation}
where \( \hat{O}_n \) projects onto the entangled observer subspace. As decoherence progresses, the norm of \( \hat{O}_n \Psi_n \) decreases, ultimately suppressing transition amplitudes in the recursive kernel.

\subsection{Recursive Duality: Lagrangian and Memory Constraints}

The recursive structure exhibits a dual constraint:

\begin{itemize}
    \item \textbf{Intra-cycle dynamics} are governed by the Lagrangian \( \mathcal{L}_n \),
    \item \textbf{Inter-cycle coherence} is enforced by an entropy balance constraint.
\end{itemize}

The resulting variational condition becomes:
\begin{equation}
\delta \mathcal{A}_{\text{total}} + \lambda_C \, \delta\left( \Delta S_{\text{fwd}} - \Delta S_{\text{mem}} \right) = 0,
\end{equation}
with:
\begin{align*}
\Delta S_{\text{fwd}} &= S[\rho_n] - S[\rho_{n-1}], \\
\Delta S_{\text{mem}} &= -\log \lambda_n, \quad \lambda_n = |\langle \Psi_{n-1} | \Psi_n \rangle|^2.
\end{align*}

Here, \( \lambda_C \) is a Lagrange multiplier enforcing memory conservation under entropy growth. This formulation constrains the evolution of admissible configurations by penalizing coherence loss beyond an allowable threshold.

\subsection{Tension Constraint and Decoherence Threshold}

To regulate coherence stress, we define the effective memory tension as:
\begin{equation}
\kappa_n := \frac{I(\phi_n, \phi_{n-1})}{\lambda_n},
\end{equation}
where \( I(\phi_n, \phi_{n-1}) = \mathrm{Tr}[\rho_n (\log \rho_n - \log \rho_{n-1})] \) is the quantum relative entropy. The recursive system must satisfy:
\begin{equation}
\kappa_n \leq \kappa_C,
\end{equation}
for a critical threshold \( \kappa_C \) corresponding to the maximum tolerable coherence strain. If violated, the system undergoes recursive collapse (e.g., dimensional rupture or supernova-class event), modeled as a discontinuous phase reset in configuration space (Appendix~\ref{appendix:D}).

This constraint prevents runaway information gain from destabilizing the recursive attractor.

\subsection{Interpretation}

The recursive variational principle unifies gravitational, informational, and thermodynamic contributions through a dual-layer constraint:

\begin{itemize}
    \item Coherent field propagation is determined by the local Lagrangian \( \mathcal{L}_n \),
    \item Recursive viability is enforced by global memory preservation encoded in the entropy constraint,
    \item Attractor convergence is determined by minimization of entropy-growth under fidelity retention,
    \item Collapse is triggered by tension overshoot, defined via the coherence-strain parameter \( \kappa_n \).
\end{itemize}

This structure yields a recursive action principle that selects for long-term coherent propagation while suppressing entropy-dominated divergences. It represents a symmetry between local dynamics and global informational stability.

