\section{Recursive Variational Principles and Symmetry of Action}
\label{sec:recursive-variational-symmetry}

\subsection{11.1 Recursive Action Formalism}

We define the recursive variational principle as:
\begin{equation}
\delta \mathcal{A}_{\text{total}} = 0 \quad \text{subject to} \quad \Delta S_{\text{fwd}} = \Delta S_{\text{mem}}
\end{equation}
This generalizes classical action principles by enforcing balance between entropy produced within a cycle and coherence lost across cycles.

The action per cycle \( \mathcal{A}_n \) is expressed in terms of the recursive Lagrangian \( \mathcal{L}_n \) (see Appendix~\ref{appendix:C}):
\begin{equation}
\mathcal{A}_n = \int dt \, \mathcal{L}_n(q_n, \dot{q}_n; q_{n-1})
\end{equation}
with configuration variables \( q_n = (a_n, \varphi_n, \lambda_n, E_n) \) and Lagrangian components for geometry, memory, decoherence, and bridge terms.

\subsection{11.2 Observer Projection and Boundary Constraint}

Observation is modeled as a boundary projection at the decoherence surface:
\begin{equation}
\delta \Psi_n \big|_{\Sigma_{\text{obs}}} = \hat{O}_n \Psi_n
\end{equation}
The operator \( \hat{O}_n \) defines the entangled observable sector of the subsystem. As coherence decays, \( \hat{O}_n \to 0 \), suppressing further contribution to recursive memory.

\subsection{11.3 Recursive Duality: Lagrangian and Memory Constraints}

Recursive evolution reflects a dual constraint structure:
\begin{itemize}
    \item The Lagrangian \( \mathcal{L}_n \) governs intra-cycle dynamics,
    \item The memory constraint governs inter-cycle propagation.
\end{itemize}
We encode this via a constrained variational condition:
\begin{equation}
\delta \mathcal{A}_{\text{total}} + \lambda_C \, \delta\left( \Delta S_{\text{fwd}} - \Delta S_{\text{mem}} \right) = 0
\end{equation}
with:
\begin{align*}
\Delta S_{\text{fwd}} &= S[\rho_n] - S[\rho_{n-1}] \\
\Delta S_{\text{mem}} &= -\log \lambda_n
\end{align*}
and where \( \lambda_n = |\langle \Psi_{n-1} | \Psi_n \rangle|^2 \) is the coherence fidelity. The multiplier \( \lambda_C \) enforces conservation of recursive memory under entropy growth.

\subsection{11.4 Tension Constraint and Decoherence Threshold}

We now impose a tension bound on recursive transitions. Define:
\begin{equation}
\kappa_n := \frac{I(\phi_n, \phi_{n-1})}{\lambda_n}
\end{equation}
where \( I(\phi_n, \phi_{n-1}) = \mathrm{Tr}[\rho_n (\log \rho_n - \log \rho_{n-1})] \) is the quantum relative entropy. The following constraint must hold:
\begin{equation}
\kappa_n \leq \kappa_C
\end{equation}
for some fixed coherence-tension threshold \( \kappa_C \). Violations trigger recursive collapse (e.g., supernovae), modeled as nonperturbative phase resets in the configuration space (see Appendix~\ref{appendix:D}).

This constraint replaces unphysical divergences in entropy cost and links coherence breakdown directly to gravitational phenomena through the memory tension budget.

\subsection{11.5 Interpretation}

The recursive variational principle formalizes evolution under dual pressures: retention of prior structure and suppression of incoherent divergence. It produces field equations that:
\begin{itemize}
    \item Propagate scalar and geometric fields under coherence filtering,
    \item Suppress transitions with excessive memory cost,
    \item Enforce attractor convergence under bounded entropy and tension,
    \item Halt recursion when \( \kappa_n > \kappa_C \), leading to structural resets.
\end{itemize}

This structure unifies geometry, information theory, and thermodynamics under a single recursive logic.

