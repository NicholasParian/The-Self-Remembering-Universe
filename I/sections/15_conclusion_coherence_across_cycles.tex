\section{Conclusion: Coherence Across Cycles}
\label{sec:conclusion}

\subsection{Summary of Contributions}

This work presents a recursive quantum cosmology framework integrating loop quantum gravity, non-Markovian decoherence, and entanglement-regulated dynamics. The evolution of the universe is governed by a transition kernel \( K(\phi, \phi') \) derived from spinfoam amplitudes and filtered through entropy- and coherence-weighted structure.

Core formal contributions include:
\begin{itemize}
    \item A derivation of the transition kernel \( K(\phi, \phi') \) from large-spin EPRL spinfoam amplitudes, embedding quantum geometry into a recursive configuration space \( \phi = (a, \varphi, \lambda, E) \).
    \item A recursive Lagrangian formalism incorporating entropy variation, memory fidelity \( \lambda_n \), and entanglement-driven decoherence kernels \( D(\tau, E) \).
    \item A fixed-point attractor state \( \Psi^*(\phi) \) defined as a recursive eigenfunction of the normalized kernel, governing long-term coherence stabilization.
    \item A thermodynamic tension–entropy constraint linking recursive tension \( \lambda_n \) to entropy flow and structural rupture thresholds (e.g., supernova events).
    \item Falsifiable predictions in gravitational wave spectra, CMB anisotropy, non-Gaussianity, polarization alignment, and recursive entropy bounds.
\end{itemize}

\subsection{Interpretative Framework}

In this model, each cosmological cycle encodes inherited structure via coherence-filtered transition amplitudes. Time is not external but emerges relationally from recursive memory propagation. The normalized kernel \( K_{\text{norm}}(\phi, \phi') \) acts as a dynamical coherence filter, suppressing decohered paths and favoring attractor convergence.

The attractor \( \Psi^*(\phi) \) represents the asymptotic configuration under recursive evolution: a memory-preserving state that satisfies entropy constraints, phase alignment, and interference stability. Relativistic speed limits are reinterpreted as coherence-collapse thresholds, beyond which recursive information flow halts.

\subsection{Empirical and Theoretical Outlook}

This framework yields distinctive, testable predictions:
\begin{itemize}
    \item Quantized suppression dips in the gravitational wave background at frequencies \( f_j \sim \sqrt{j(j+1)} / 2\pi \ell_{\text{Pl}} \),
    \item Suppressed CMB angular power at low multipoles \( \ell < 30 \) due to Gaussian filtering in the kernel,
    \item Recursive non-Gaussianity \( f_{\text{NL}}^{\text{rec}} \) driven by coherence fidelity \( \lambda_n \) and entanglement scale \( E \),
    \item Parity-violating EB-mode polarization localized in large-scale voids, seeded by entanglement asymmetry,
    \item Bounded entropy growth governed by thermodynamic compensation between information gain and radiative entropy loss.
\end{itemize}

These predictions are distinguishable from inflationary, ekpyrotic, or CCC-based models, and are within the sensitivity range of \textbf{LISA}, \textbf{LiteBIRD}, \textbf{CMB-S4}, \textbf{SKA}, and \textbf{Euclid}.

\subsection{Open Problems and Research Directions}

Key unresolved challenges include:
\begin{itemize}
    \item Completion of the recursive Euler–Lagrange derivation including all variational couplings,
    \item Extension of the kernel with quantum group corrections and topology change contributions,
    \item Dynamical modeling of the observer operator \( \hat{O}_n \) and its feedback on decoherence structure,
    \item High-fidelity numerical simulations of recursive attractor dynamics in minisuperspace,
    \item Empirical constraints on recursive string tension \( \lambda_n \) via gravitational wave and entropy-bound observations.
\end{itemize}

\subsection{Closing Perspective}

This work introduces a coherent, mathematically complete framework for cosmological evolution grounded in quantum geometry, memory theory, and entanglement dynamics. The recursive action principle, transition kernel, and attractor convergence structure together define a new paradigm for understanding the universe as a memory-preserving system. If validated, this model may unify geometry, thermodynamics, and information under a single recursive physical law.
