\section{Recursive Transition Kernel \( K(\phi,\phi') \)}
\label{sec:kernel}

The transition kernel \( K(\phi,\phi') \equiv K(a,\varphi,\lambda,E; a',\varphi',\lambda',E') \) governs recursive evolution of quantum states across cosmological cycles. It defines the complex amplitude for transitioning between configurations \( \phi \) and \( \phi' \), incorporating memory fidelity, entropic divergence, and phase coherence. This kernel is formally derived in Appendix~\ref{appendix:C2} from the large-spin asymptotics of the EPRL spinfoam model.

The kernel acts as:
\begin{itemize}
    \item A \textbf{coherence filter} selecting phase- and geometry-aligned configurations,
    \item An \textbf{entropy gate} penalizing misaligned transitions via quantum relative entropy,
    \item A \textbf{geometric propagator} enforcing bridge continuity across ERB boundaries,
    \item A \textbf{tension regulator} suppressing unphysical information spikes beyond coherence stress bounds.
\end{itemize}

\subsection{Normalized Recursive Operator}
\label{subsec:normalized-kernel}

Recursive evolution proceeds via:
\begin{equation}
\Psi_{n+1}(\phi) = \int d\phi' \, K_{\text{norm}}(\phi, \phi') \Psi_n(\phi'),
\end{equation}
where:
\begin{align}
K_{\text{norm}}(\phi, \phi') &= \frac{K_{\text{eff}}(\phi, \phi')}{Z(\phi')}, \\
Z(\phi') &= \int d\phi \, K_{\text{eff}}(\phi, \phi').
\end{align}
This normalization enforces probability conservation and supports the convergence proof in Appendix~\ref{appendix:C9}.

\subsection{Effective Kernel Structure}
\label{subsec:effective-kernel}

The unnormalized kernel is:
\begin{equation}
K_{\text{eff}}(\phi, \phi') = \exp\left[i S_{\text{ERB}}(\phi, \phi') - \lambda_S I(\phi, \phi') + \lambda_C C(\phi, \phi') \right] \cdot \mathcal{F}_C(\phi, \phi'),
\end{equation}
with:
\begin{itemize}
    \item \( S_{\text{ERB}}(\phi, \phi') \): ER bridge action between configurations (Appendix~C.3),
    \item \( I(\phi, \phi') := \mathrm{Tr}[\rho_\phi (\log \rho_\phi - \log \rho_{\phi'})] \): quantum relative entropy,
    \item \( C(\phi, \phi') := \langle \Psi_{n-1} | \mathcal{P}_{\phi,\phi'} | \Psi_{n-1} \rangle \): prior-cycle coherence overlap.
\end{itemize}

\subsection{Coherence Filter \( \mathcal{F}_C(\phi,\phi') \)}
\label{subsec:coherence-filter}

The Gaussian filter restricts transitions to phase- and curvature-aligned configurations:
\[
\mathcal{F}_C(\phi, \phi') = \exp\left[
    -\frac{(a - a')^2}{2\sigma_a^2}
    -\frac{(\varphi - \varphi')^2}{2\sigma_\varphi^2}
    -\frac{(\theta(\lambda) - \theta(\lambda'))^2}{2\sigma_\theta^2}
    -\frac{(E - E')^2}{2\sigma_E^2}
\right]
\]
where:
\begin{itemize}
    \item \( \theta(\lambda) \): phase function mapping memory fidelity to interference angle,
    \item \( \sigma_a \): geometric width (typically Planck scale),
    \item \( \sigma_\varphi \sim \lambda^{-1/2} \): scalar coherence width,
    \item \( \sigma_\theta \sim \lambda^{-1} \): phase sensitivity scaling,
    \item \( \sigma_E \sim E^{-1} \): entanglement resolution bound.
\end{itemize}

\subsection{Kernel Derivation}
\label{subsec:kernel-derivation}

\paragraph{Canonical LQC (Minisuperspace Constraint):}
\[
\hat{H}_{\text{LQC}}\Psi(a,\varphi) = \left[
    -\frac{\hbar^2}{2}\frac{\partial^2}{\partial a^2} + V_{\text{eff}}(a,\varphi)
\right]\Psi(a,\varphi) = 0
\]
Bounce symmetry conditions are imposed at minimal scale:
\[
\Psi(a_{\text{min}}^-, \varphi) = \Psi(a_{\text{min}}^+, \varphi), \quad
\partial_a\Psi|_{a_{\text{min}}^-} = \partial_a\Psi|_{a_{\text{min}}^+}
\]

\paragraph{Covariant LQG (Spinfoam Path Integral):}
As shown in Appendix~\ref{appendix:C2}, we construct:
\[
K(\phi, \phi') \sim \exp\left[i S_{\text{ERB}}(\phi, \phi')\right] \cdot \mathcal{F}_C(\phi, \phi')
\]
with LQG mappings:
\begin{itemize}
    \item \( a \): face area \( A_f \sim \sqrt{j(j+1)} \),
    \item \( \varphi \): scalar field node labels,
    \item \( \lambda \): phase fidelity between attractor states,
    \item \( E \): ERB throat entanglement.
\end{itemize}

\subsection{Observables and Kernel Parameters}
\label{subsec:observables-and-kernel-parameters}

\begin{table}[H]
\centering
\begin{tabular}{lll}
\toprule
\textbf{Kernel Parameter} & \textbf{Physical Mapping} & \textbf{Observable Signature} \\
\midrule
\( \sigma_\varphi \sim \lambda^{-1/2} \) & Field alignment width & Non-Gaussianity \( f_{\text{NL}} \sim \sigma_\varphi^{-1} \) \\
\( \sigma_E \sim E^{-1} \) & Entanglement resolution & EB-mode polarization alignment \\
\( \lambda \) & Memory fidelity & Overlap decay rate \( \partial_n \lambda \) \\
\( j_0 \) & Dominant spin & GW spectrum peak near \( f_j \sim \sqrt{j_0(j_0+1)} \) \\
\bottomrule
\end{tabular}
\caption{Mapping kernel parameters to observable features.}
\end{table}

\subsection{Tension and Boundary Constraints}
\label{subsec:tension-constraints}

To enforce physical viability of transitions:
\[
W_{\text{constraints}} = \exp\left[
    -\lambda_A\left(\frac{\ell_{\text{Pl}}^2}{A(\phi,\phi')}\right)^{\alpha} 
    - \lambda_S\left(\frac{4G\hbar\, S_{\text{rec}}}{A(\phi,\phi')}\right)^{\beta}
\right]
\]

Additional tension-based suppression:
\[
W_T(\phi,\phi') = \exp\left[
  -\lambda_T \left( \frac{I(\phi, \phi')}{\lambda} - \kappa_C \right)^2
\right]
\]
where \( \kappa_C \) is the coherence rupture threshold. Excessive divergence triggers suppressed branching or collapse (see Section~\ref{sec:mathematical-framework} and Appendix~\ref{appendix:D}).

\subsection{Numerical Strategy}
\label{subsec:numerical}

Kernel evolution is simulated via:
\begin{itemize}
    \item Monte Carlo integration over dominant spin sectors,
    \item Crank–Nicolson time evolution for minisuperspace modes,
    \item Fidelity-weighted entropy decay monitoring,
    \item Attractor alignment check via \( \lambda_n \to 1 \) and entropy stabilization.
\end{itemize}
Full numerical implementation is detailed in Appendix~\ref{appendix:C5}.

\bigskip
\noindent
\textit{Note: \( K(\phi, \phi') \) is a complex amplitude, not a classical transition probability. Recursive coherence emerges through constructive interference across cycles.}
