\section{Recursive Transition Kernel \( K(\phi,\phi') \)}
\label{sec:kernel}

The transition kernel \( K(\phi,\phi') \equiv K(a,\varphi,\lambda,E; a',\varphi',\lambda',E') \) governs recursive evolution of quantum states across cosmological cycles. It defines the probability amplitude for transitions between configurations \( \phi \) and \( \phi' \), incorporating memory fidelity, coherence filtering, and entropic admissibility.

The kernel acts as:
\begin{itemize}
    \item A \textbf{coherence filter} selecting phase- and geometry-aligned configurations,
    \item An \textbf{entropy gate} preserving recursive memory under quantum relative entropy,
    \item A \textbf{geometric bridge} enforcing spinfoam-mapped structure across ERB boundaries,
    \item A \textbf{tension regulator} penalizing unphysical transitions driven by excessive information gain.
\end{itemize}

\subsection{Normalized Recursive Operator}
\label{subsec:normalized-kernel}

The evolution operator acts on recursive states via:
\begin{equation}
\Psi_{n+1}(\phi) = \int d\phi' \, K_{\text{norm}}(\phi, \phi') \Psi_n(\phi'),
\end{equation}
where:
\begin{align}
K_{\text{norm}}(\phi, \phi') &= \frac{K_{\text{eff}}(\phi, \phi')}{Z(\phi')}, \\
Z(\phi') &= \int d\phi \, K_{\text{eff}}(\phi, \phi').
\end{align}
The normalization ensures conservation of probability and supports the convergence proof in Appendix~\ref{appendix:C9}.

\subsection{Effective Kernel Structure}
\label{subsec:effective-kernel}

The unnormalized effective kernel is defined as:
\begin{equation}
K_{\text{eff}}(\phi, \phi') = \exp\left[-\lambda_S I(\phi, \phi') + \lambda_C C(\phi, \phi') \right] \cdot \mathcal{F}_C(\phi, \phi'),
\end{equation}
with:
\begin{itemize}
    \item \( I(\phi, \phi') := \mathrm{Tr}[\rho_\phi (\log \rho_\phi - \log \rho_{\phi'})] \): quantum relative entropy,
    \item \( C(\phi, \phi') := \Psi_{n-1}^*(\phi) \Psi_{n-1}(\phi') \): coherence overlap from prior cycle,
    \item \( \mathcal{F}_C(\phi, \phi') \): Gaussian envelope filter defined below.
\end{itemize}

\subsection{Coherence Filter \( \mathcal{F}_C(\phi,\phi') \)}
\label{subsec:coherence-filter}

The coherence filter enforces alignment in geometric and entanglement variables:
\[
\mathcal{F}_C(\phi, \phi') = \exp\left[
    -\frac{(a - a')^2}{2\sigma_a^2}
    -\frac{(\varphi - \varphi')^2}{2\sigma_\varphi^2}
    -\frac{(\theta(\lambda) - \theta(\lambda'))^2}{2\sigma_\theta^2}
    -\frac{(E - E')^2}{2\sigma_E^2}
\right]
\]
\begin{itemize}
    \item \( \theta(\lambda) \): phase function associated with recursive tension coherence,
    \item \( \sigma_{a,\varphi,\theta,E} \): filter widths controlling transition selectivity.
\end{itemize}
This form enforces constructive interference and continuity in curvature and entanglement structure.

\subsection{Kernel Derivation}
\label{subsec:kernel-derivation}

\paragraph{Canonical LQC (Minisuperspace Constraint):}
\[
\hat{H}_{\text{LQC}}\Psi(a,\varphi) = \left[
    -\frac{\hbar^2}{2}\frac{\partial^2}{\partial a^2} + V_{\text{eff}}(a,\varphi)
\right]\Psi(a,\varphi) = 0
\]
Bounce symmetry is enforced across discrete steps:
\[
\Psi(a_{\text{min}}^-, \varphi) = \Psi(a_{\text{min}}^+, \varphi), \quad
\partial_a\Psi|_{a_{\text{min}}^-} = \partial_a\Psi|_{a_{\text{min}}^+}
\]

\paragraph{Covariant LQG (Spinfoam Path Integral):}
As derived in Appendix~\ref{appendix:C2}, the kernel is obtained from large-spin asymptotics of the EPRL amplitude:
\[
K(\phi, \phi') \sim \exp\left[i S_{\text{ERB}}(\phi, \phi')\right] \cdot \mathcal{F}_C(\phi, \phi')
\]
with configuration variables mapped from:
\begin{itemize}
    \item \( a \): face areas \( A_f \sim \sqrt{j(j+1)} \),
    \item \( \varphi \): scalar field boundary labels,
    \item \( \lambda \): phase stability across cycles,
    \item \( E \): ERB throat area (entanglement eigenvalue).
\end{itemize}

\subsection{Observables and Kernel Parameters}
\label{subsec:observables-and-kernel-parameters}

\begin{table}[H]
\centering
\begin{tabular}{lll}
\toprule
\textbf{Kernel Parameter} & \textbf{Physical Mapping} & \textbf{Observable Signature} \\
\midrule
\( \sigma_\varphi \) & Field alignment width & CMB non-Gaussianity \( f_{\text{NL}} \sim \sigma_\varphi^{-1} \) \\
\( \sigma_E \) & Entanglement coherence width & EB-mode polarization alignment \\
\( \langle \Psi_n | \Psi_{n-1} \rangle \) & Inter-cycle fidelity & Entropy decay rate \( \dot{S}_n \) \\
\( j_0 \) & Dominant spin scale & GW spectrum peak frequency \\
\bottomrule
\end{tabular}
\caption{Mapping between kernel structure and observable cosmological features.}
\end{table}

\subsection{Tension and Boundary Constraints}
\label{subsec:tension-constraints}

Transitions are regulated by soft constraints enforcing minimum area and bounded entropy flow:
\[
W_{\text{constraints}} = \exp\left[
    -\lambda_A\left(\frac{\ell_{\text{Pl}}^2}{A(\phi,\phi')}\right)^{\alpha} 
    - \lambda_S\left(\frac{4G\hbar\, S_{\text{rec}}}{A(\phi,\phi')}\right)^{\beta}
\right]
\]

Tension regulation is incorporated through entropy divergence suppression:
\[
W_T(\phi,\phi') = \exp\left[
  -\lambda_T \left( \frac{I(\phi, \phi')}{\lambda} - \kappa_C \right)^2
\right]
\]
with:
\begin{itemize}
    \item \( \kappa_C \): critical coherence stress threshold (from broken dimension/string logic),
    \item \( \lambda \): memory fidelity from the current cycle.
\end{itemize}
This term is absorbed into the effective kernel \( K_{\text{eff}} \) and governs supernova-like rupture transitions under excessive memory strain (see Section~\ref{sec:recursive-variational} and Appendix~\ref{appendix:D}).

\subsection{Numerical Strategy}
\label{subsec:numerical}

Monte Carlo integration over dominant spin sectors is used to approximate the kernel. Adaptive Crank–Nicolson evolution integrates the LQC sector. Entropy-aware memory filtering enforces convergence to the attractor state \( \Psi^*(\phi) \). See Appendix~C.5 for full numerical details.

\vspace{0.5em}
\noindent
\textit{For a formal proof of convergence under recursive kernel evolution, see Appendix~\ref{appendix:C9}.}
