\section*{Appendix C\\Recursive Lagrangian Dynamics and First-Principles Kernel Derivation}
\addcontentsline{toc}{section}{Appendix C: Recursive Lagrangian Dynamics and First-Principles Kernel Derivation}
\label{appendix:C}

\subsection*{Kinematic Structure of Recursive Configuration Space}

The configuration space governing recursive quantum cosmology is defined as:
\[
\phi = (a, \varphi, \lambda, E)
\]
where:
\begin{itemize}
  \item \( a \): scale factor, treated as a discrete geometric degree of freedom in LQC,
  \item \( \varphi \): scalar field amplitude(s), encoding matter configuration on the spatial hypersurface,
  \item \( \lambda \): fidelity eigenvalue, interpreted as a dimensionless tension variable across cycles,
  \[
  \lambda := |\langle \Psi_{n-1} | \Psi_n \rangle|^2 \in [0,1]
  \]
  Coherence failure at low \( \lambda \) corresponds to structural breakdown, such as supernova-like events resulting from recursive instability.
  \item \( E \): entanglement eigenvalue, defined as the square root of the von Neumann entropy of the reduced density matrix across the ERB,
  \[
  E := \sqrt{S(\rho_{\text{red}})} = \sqrt{-\mathrm{Tr}(\rho_{\text{red}} \log \rho_{\text{red}})}
  \]
  representing the flux of entanglement coherence across the inter-cycle boundary.
\end{itemize}

This four-component structure is minimal yet sufficient to encode:
\begin{enumerate}
    \item Quantum geometry (via \( a \)),
    \item Scalar field dynamics and boundary states (via \( \varphi \)),
    \item Inter-cycle coherence and memory propagation (via \( \lambda \)),
    \item Internal entanglement structure and decoherence scale (via \( E \)).
\end{enumerate}

Each cycle \( n \) evolves a configuration \( \phi_n \) subject to recursive update equations of the form:
\[
\Psi_n(\phi) = \int K(\phi, \phi') \Psi_{n-1}(\phi') \, e^{i S_{\text{ERB}}(\phi, \phi')} \, d\phi'
\]
The transition kernel \( K(\phi, \phi') \) and entropic action \( S_{\text{ERB}} \) are defined in Sections~\ref{sec:kernel} and~\ref{sec:recursive-action-formal}.

This structure defines the configuration space on which recursive dynamics, attractor behavior, and observational signatures are evaluated.

\subsection*{Kernel Derivation from Spinfoam Amplitudes}
\label{appendix:C2}

The recursive transition kernel \( K(\phi, \phi') \) defines the amplitude for evolution between configuration states \(\phi_n\) and \(\phi_{n+1}\), incorporating quantum geometry, entanglement memory, and coherence filtering. We derive this kernel from the covariant loop quantum gravity (LQG) formalism using the large-spin asymptotics of the EPRL spinfoam model~\cite{barrett2009asymptotic, engle2008lqg}.

\subsubsection*{C.2.1 Spinfoam Construction}

The spinfoam transition amplitude over a 2-complex \(\mathcal{C}\) is:
\[
Z(\mathcal{C}) = \sum_{j_f, \iota_v} \prod_f (2j_f + 1) \prod_v A_v(j_f, \iota_v),
\]
where:
\begin{itemize}
  \item \( j_f \): Spin label on face \( f \), encoding quantized area,
  \item \( \iota_v \): Livine--Speziale intertwiner at vertex \( v \),
  \item \( A_v \): Vertex amplitude, peaking on Regge geometry in the semiclassical limit.
\end{itemize}

\subsubsection*{C.2.2 Configuration Embedding}

We define the recursive configuration state \(\phi = (a, \varphi, \lambda, E)\) with the following mappings:

\begin{table}[H]
\centering
\begin{tabular}{lll}
\toprule
\textbf{Component} & \textbf{LQG Mapping} & \textbf{Interpretation} \\
\midrule
\( a \) & \( A_f = 8\pi \gamma \ell_{\text{Pl}}^2 \sqrt{j(j+1)} \) & Discrete scale factor from face area \\
\( \varphi \) & Node scalar label or insertion at boundary vertex & Field amplitude on 3-slice \\
\( \lambda \) & Phase stability across cycles & Recursive coherence/tension (failure triggers structural collapse) \\
\( E \) & Dual ERB area from shared face sets & Entanglement flux boundary eigenvalue \\
\bottomrule
\end{tabular}
\caption{Mapping of LQG spinfoam data to recursive configuration vector components.}
\end{table}

\subsubsection*{C.2.3 Asymptotic Kernel Structure}

In the large-spin limit, the spinfoam amplitude asymptotically reduces to:
\[
K(\phi, \phi') \sim \exp\left[i S_{\text{ERB}}(\phi, \phi')\right] \cdot \mathcal{F}(\phi, \phi'),
\]
where:
\begin{itemize}
  \item \( S_{\text{ERB}}(\phi, \phi') \) is the entropic action across the Einstein--Rosen bridge,
  \item \( \mathcal{F}(\phi, \phi') \) is a Gaussian envelope function filtering incoherent paths.
\end{itemize}

\subsection*{Entropic Bridge Action}

We define the ERB action as a quadratic functional over configuration differences:
\[
S_{\text{ERB}}(\phi, \phi') = \alpha_a (a - a')^2 + \alpha_\varphi (\varphi - \varphi')^2 + \alpha_E (E - E')^2 + \alpha_\lambda (\lambda - \lambda')^2,
\]
with coupling constants \(\alpha_a, \alpha_\varphi, \alpha_E, \alpha_\lambda > 0\) encoding resistance to change across cycles. These constants may be constrained by observational data or entropy minimization principles.

\subsection*{Coherence Filtering Function}

We define:
\[
\mathcal{F}(\phi, \phi') = \exp\left[-\frac{(a - a')^2}{2\sigma_a^2} - \frac{(\varphi - \varphi')^2}{2\sigma_\varphi^2} - \frac{(E - E')^2}{2\sigma_E^2}\right],
\]
where \(\sigma_{a,\varphi,E}\) are coherence tolerance widths. This structure mimics a Gaussian filter derived from variation in classical Regge action near semiclassical geometries, enforcing interference alignment.

This completes the derivation of the kernel \( K(\phi, \phi') \) from first principles in covariant loop quantum gravity, anchored to recursive memory-preserving dynamics.
\subsection*{Numerical Implementation}
\label{appendix:C5}

Numerical simulation of recursive quantum cosmology requires integration of LQC dynamics, spinfoam amplitude evaluation, and memory kernel propagation. This section outlines the key components and computational priorities.

\subsubsection*{Spinfoam Monte Carlo Sampling}

The transition kernel \( K(\phi, \phi') \) is approximated using a spin-sum Monte Carlo method over boundary 2-complexes:
\begin{equation}
\langle K \rangle = \frac{1}{N} \sum_{i=1}^N \prod_v A_v^{(i)} \, e^{i S_{\text{ERB}}^{(i)}}
\end{equation}
where each term samples a boundary spin configuration \( \{j_f^{(i)}, \iota_v^{(i)}\} \), with:
\begin{itemize}
  \item Importance sampling near dominant spin scale \( j_0 \sim A_{\text{ERB}}/8\pi\gamma\ell_P^2 \),
  \item Filtered by a coherence envelope \( \mathcal{F}(a, a', \phi, \phi') \).
\end{itemize}
Parallel tempering is used across spin sectors to improve convergence in the presence of interference nodes.

\subsubsection*{Discrete LQC Evolution}

The LQC minisuperspace Hamiltonian is:
\begin{equation}
\hat{H}_{\text{LQC}} = -\frac{3\pi G}{2} \frac{p_a^2}{a} + a^3 V(\phi)
\end{equation}
This is solved using adaptive grids near the bounce and a Crank–Nicolson semi-implicit method for wavefunction evolution:
\begin{equation}
\Psi_{n+1}(a,\phi) = \hat{U}_{\text{LQC}} \Psi_n(a,\phi)
\end{equation}
Normalization is enforced:
\begin{equation}
\|\Psi_n\|^2 = \int da \, d\phi \, \mu(a) |\Psi_n(a,\phi)|^2
\end{equation}
where \( \mu(a) \) is the LQC measure.

\subsubsection*{Memory Kernel Discretization}

The decoherence kernel is implemented via finite-memory convolution:
\begin{equation}
D_{\text{disc}}(\tau_m) = \gamma(S) \, e^{-\tau_m/\tau_c(S)} \cos[\omega_0(S) \tau_m] \, \Delta \tau
\end{equation}
where:
\begin{itemize}
  \item \( \tau_m \): discretized delay index (\( \tau_m = m \Delta \tau \)),
  \item \( \gamma(S), \tau_c(S), \omega_0(S) \): dynamically updated from entropy profile \( S(t) \).
\end{itemize}
The memory window \( T_{\text{mem}} \sim 5\tau_c \) is sufficient for convergence.

\subsubsection*{Fidelity and Attractor Tracking}

Inter-cycle fidelity is computed as:
\begin{equation}
M_n = \frac{|\langle \Psi_n | \Psi_{n-1} \rangle|^2}{\|\Psi_n\|^2}
\end{equation}
The coherence fitness functional \( \mathcal{F}_n \) is logged at each step. Convergence toward the attractor \( \Psi^*(\phi) \) is assessed by:
\begin{equation}
D_n = \| \Psi_{n+1} - \Psi_n \|_{L^2}, \quad D_n \to 0
\end{equation}

\subsubsection*{Implementation Notes}

\begin{itemize}
  \item Field space is truncated to a finite box \( \phi \in [-\phi_{\max}, \phi_{\max}] \) with absorbing boundary conditions.
  \item Numerical instabilities at large \( j_f \) are regulated using adaptive spin cutoffs.
  \item Codebase can be structured using TensorFlow or Julia with GPU acceleration for spin amplitude evaluation.
\end{itemize}

\subsection*{Complete Recursive Action Functional}
\label{appendix:C6}

The recursive evolution of the universe is governed by a total action functional spanning geometry, quantum coherence, and memory propagation. We define:
\begin{equation}
\mathcal{A}_{\text{total}} = \sum_{n} \left[ \mathcal{A}_{\text{EH}}^{(n)} + \mathcal{A}_{\text{ERB}}^{(n)} + \mathcal{A}_{\text{mem}}^{(n)} \right]
\end{equation}
where each component encodes a distinct physical contribution per cycle.

\subsubsection*{Einstein–Hilbert Sector}

The gravitational term is given by the standard action over 4D spacetime:
\begin{equation}
\mathcal{A}_{\text{EH}}^{(n)} = \int_{\mathcal{M}_n} d^4x \, \sqrt{-g} \left( \frac{R}{2\kappa} - \Lambda + \mathcal{L}_{\text{matter}} \right)
\end{equation}
where:
\begin{itemize}
  \item \( R \): Ricci scalar curvature,
  \item \( \kappa = 8\pi G \),
  \item \( \mathcal{L}_{\text{matter}} \): scalar field lagrangian, \( \mathcal{L}_{\phi} = -\frac{1}{2} g^{\mu\nu} \partial_\mu \phi \partial_\nu \phi - V(\phi) \),
  \item \( \mathcal{M}_n \): manifold corresponding to cycle \( n \).
\end{itemize}

\subsubsection*{Einstein–Rosen Bridge Sector}

The bridge action encodes thermodynamic and informational flow across cycles:
\begin{equation}
\mathcal{A}_{\text{ERB}}^{(n)} = \frac{A_n}{4G} + i\lambda_E I(\phi_n, \phi_{n-1})
\end{equation}
where:
\begin{itemize}
  \item \( A_n \): minimal area of the ER bridge at cycle \( n \),
  \item \( I(\phi_n, \phi_{n-1}) = \mathrm{Tr}\left[ \rho_n (\log \rho_n - \log \rho_{n-1}) \right] \): quantum relative entropy (see Appendix E),
  \item \( \lambda_E \): coherence penalty coupling parameter.
\end{itemize}

This term ensures that only memory-compatible transitions (low \( I \)) dominate the kernel amplitude.

\subsubsection*{Memory Entropy Sector}

The memory term penalizes loss of coherence between cycles:
\begin{equation}
\mathcal{A}_{\text{mem}}^{(n)} = -\beta^{-1} \log \left( |\langle \Psi_n | \Psi_{n-1} \rangle| + \epsilon \right)
\end{equation}
with:
\begin{itemize}
  \item \( \beta \): inverse memory temperature, modulating entropy sensitivity,
  \item \( \epsilon \): regularization parameter to ensure finite penalty near orthogonality.
\end{itemize}

This term reflects the system’s tendency to preserve quantum overlap and suppress transition to decohered branches.

\subsubsection*{Variational Principle}

The full recursive dynamics are determined by the condition:
\begin{equation}
\delta \mathcal{A}_{\text{total}} = 0
\end{equation}
subject to:
\begin{equation}
\Delta S_{\text{fwd}} = \Delta S_{\text{mem}}
\end{equation}
This enforces balance between forward entropy increase and backward memory retention, and ensures convergence toward coherence-stabilized attractor states.

\subsection*{Recursive Attractor Definition and Variational Structure}
\label{appendix:C7}

We define the recursive attractor state \( \Psi^*(\phi) \) as the fixed point of the filtered transition kernel:
\begin{equation}
\Psi^*(\phi) = \int d\phi' \, K_{\text{norm}}(\phi, \phi') \Psi^*(\phi'),
\end{equation}
with the normalized effective kernel:
\begin{equation}
K_{\text{norm}}(\phi, \phi') = \frac{K_{\text{eff}}(\phi, \phi')}{Z(\phi')}, \quad Z(\phi') := \int d\phi \, K_{\text{eff}}(\phi, \phi').
\end{equation}

The unnormalized kernel includes entropy divergence, coherence inheritance, and Gaussian filtering:
\begin{equation}
K_{\text{eff}}(\phi, \phi') = \exp\left[-\lambda_S I(\phi, \phi') + \lambda_C C(\phi, \phi') \right] \cdot \mathcal{F}_C(\phi, \phi'),
\end{equation}
where:
\begin{itemize}
  \item \( I(\phi, \phi') = \mathrm{Tr}[\rho_\phi (\log \rho_\phi - \log \rho_{\phi'})] \): quantum relative entropy between reduced boundary states,
  \item \( C(\phi, \phi') = \Psi_{n-1}^*(\phi) \Psi_{n-1}(\phi') \): coherence overlap from the prior cycle,
  \item \( \mathcal{F}_C(\phi, \phi') \): Gaussian coherence filter defined in Appendix~C.4.
\end{itemize}

\subsubsection*{Variational Principle}

The attractor can also be characterized as the minimizer of a coherence-weighted entropy functional:
\begin{equation}
\Psi^*(\phi) = \arg\min_{\Psi} \left\{ \lambda_S S(\rho_\Psi) + \lambda_T \lambda^2 - \lambda_C |\langle \Psi | \Psi_{n-1} \rangle|^2 \right\},
\end{equation}
subject to normalization \( \langle \Psi | \Psi \rangle = 1 \), where:
\begin{itemize}
  \item \( S(\rho_\Psi) \): von Neumann entropy of the reduced state traced over \( a \) and \( E \),
  \item \( \lambda \): recursive tension from phase mismatch,
  \item \( |\langle \Psi | \Psi_{n-1} \rangle|^2 \): memory fidelity from the prior cycle.
\end{itemize}

\subsubsection*{Interpretation}

The attractor satisfies three joint constraints:
\begin{enumerate}
  \item \textbf{Fixed-point evolution:} \( \Psi^* \) is invariant under recursive application of the transition kernel,
  \item \textbf{Entropy–coherence tradeoff:} Attractor minimizes disorder while retaining memory,
  \item \textbf{Interference selection:} Filtering enforces curvature and phase alignment between configurations.
\end{enumerate}

These conditions define \( \Psi^*(\phi) \) as a self-consistent recursive eigenstate stabilizing quantum memory across cycles.

\subsubsection*{Numerical Realization}

The attractor is computed iteratively via:
\begin{align}
\Psi_{n+1}(\phi) &= \int d\phi' \, K_{\text{norm}}(\phi, \phi') \Psi_n(\phi') \\
\|\Psi_n\|^2 &= \int d\phi \, |\Psi_n(\phi)|^2 = 1 \\
D_n &= \| \Psi_{n+1} - \Psi_n \|_{L^2} \to 0.
\end{align}

Convergence is guaranteed by the contraction property of \( K_{\text{norm}} \) (see Appendix~C.9). The resulting attractor defines a coherent fixed point of recursive cosmological evolution.
\subsection*{Thermodynamic Variational Constraint and Entropic Compensation}
\label{appendix:C8}

We impose a new variational constraint motivated by thermodynamic consistency:
\begin{equation}
\Delta S_{\text{gain}} + \Delta S_{\text{rad}} = \Delta S_{\text{exp}}
\end{equation}
Here:
\begin{itemize}
  \item \( \Delta S_{\text{gain}} \): entropy reduction due to information sharpening (coherence gain),
  \item \( \Delta S_{\text{exp}} \): entropy dilution from expansion,
  \item \( \Delta S_{\text{rad}} \): compensatory entropy flux via Hawking radiation.
\end{itemize}

This condition is enforced via an additional Lagrange multiplier term in the recursive action:
\begin{equation}
\delta \mathcal{A}_{\text{total}} + \lambda_T \, \delta\left( \Delta S_{\text{gain}} + \Delta S_{\text{rad}} - \Delta S_{\text{exp}} \right) = 0
\end{equation}

We interpret \( \lambda_T \) as a thermodynamic tension coupling, related to maximum coherence propagation allowed per cycle. This supplements the entropy-memory balance condition from Section~\ref{appendix:C6} and ties memory propagation to a radiative backreaction mechanism.

Supernovae are modeled as coherence rupture events, where excessive tension (information gain beyond threshold) causes string failure. The number of dimensions whose tension threshold is exceeded determines the type of collapse and radiative signature.

This constraint will influence attractor convergence and kernel admissibility, and may yield observable signatures in late-time entropy gradients and gravitational wave backgrounds.
\subsection*{Attractor Convergence Proof via Contraction Mapping}
\label{appendix:C9}

We now prove that the recursive update equation for the attractor state \( \Psi^*(\phi) \) defines a contraction mapping on a suitable Hilbert space. This guarantees convergence from arbitrary initial states to a unique, stable attractor.

\subsubsection*{Recursive Operator Definition}

Let the recursive update operator be:
\begin{equation}
\mathcal{T}[\Psi](\phi) := \int d\phi' \, K_{\text{norm}}(\phi, \phi') \Psi(\phi'),
\end{equation}
where the normalized kernel is:
\begin{equation}
K_{\text{norm}}(\phi, \phi') := \frac{K_{\text{eff}}(\phi, \phi')}{Z(\phi')}, \quad Z(\phi') = \int d\phi \, K_{\text{eff}}(\phi, \phi').
\end{equation}
The effective kernel is defined as:
\begin{equation}
K_{\text{eff}}(\phi, \phi') = \exp\left[ -\lambda_S I(\phi, \phi') + \lambda_C C(\phi, \phi') \right] \cdot \mathcal{F}_C(\phi, \phi'),
\end{equation}
where:
\begin{itemize}
    \item \( I(\phi, \phi') \): quantum relative entropy between reduced states,
    \item \( C(\phi, \phi') \): coherence overlap with prior-cycle memory,
    \item \( \mathcal{F}_C(\phi, \phi') \): Gaussian filter in configuration space.
\end{itemize}

\subsubsection*{Function Space and Norm}

Let \( \mathcal{H} = L^2(\mathcal{C}) \) denote the Hilbert space of square-integrable functions over configuration space \( \mathcal{C} \), with norm:
\begin{equation}
\| \Psi \|^2 = \int d\phi \, |\Psi(\phi)|^2.
\end{equation}
We aim to show that \( \mathcal{T} \) is a strict contraction:
\begin{equation}
\exists \, \gamma \in (0,1) \quad \text{such that} \quad \|\mathcal{T}[\Psi_1] - \mathcal{T}[\Psi_2]\| \leq \gamma \|\Psi_1 - \Psi_2\|.
\end{equation}

\subsubsection*{Proof of Contraction}

Let \( \Delta \Psi := \Psi_1 - \Psi_2 \). We have:
\begin{align}
\left| \mathcal{T}[\Psi_1](\phi) - \mathcal{T}[\Psi_2](\phi) \right|^2
&= \left| \int d\phi' \, K_{\text{norm}}(\phi, \phi') \Delta \Psi(\phi') \right|^2 \\
&\leq \left( \int d\phi' \, K_{\text{norm}}(\phi, \phi') \right)
\left( \int d\phi' \, K_{\text{norm}}(\phi, \phi') |\Delta \Psi(\phi')|^2 \right),
\end{align}
where the inequality follows from Cauchy--Schwarz. Using \( \int d\phi \, K_{\text{norm}}(\phi, \phi') = 1 \), we integrate both sides:
\begin{align}
\| \mathcal{T}[\Psi_1] - \mathcal{T}[\Psi_2] \|^2
&\leq \iint d\phi \, d\phi' \, K_{\text{norm}}(\phi, \phi') |\Delta \Psi(\phi')|^2 \\
&= \int d\phi' \, |\Delta \Psi(\phi')|^2 \left[ \int d\phi \, K_{\text{norm}}(\phi, \phi') \right] \\
&= \| \Psi_1 - \Psi_2 \|^2.
\end{align}

Strict contraction holds because the kernel includes:
\begin{itemize}
    \item \( \mathcal{F}_C(\phi, \phi') < 1 \) for \( \phi \ne \phi' \),
    \item \( I(\phi, \phi') > 0 \) for distinguishable states,
    \item Normalization ensures bounded operator norm.
\end{itemize}
Therefore:
\begin{equation}
\| \mathcal{T}[\Psi_1] - \mathcal{T}[\Psi_2] \| < \| \Psi_1 - \Psi_2 \|,
\end{equation}
and \( \mathcal{T} \) is a contraction on \( \mathcal{H} \).

\subsubsection*{Conclusion}

By the Banach fixed point theorem, the operator \( \mathcal{T} \) has a unique fixed point \( \Psi^*(\phi) \in \mathcal{H} \), and any initial state \( \Psi_0 \in \mathcal{H} \) converges under recursive iteration:
\[
\Psi_{n+1} = \mathcal{T}[\Psi_n] \quad \Rightarrow \quad \Psi_n \to \Psi^*(\phi).
\]
This completes the mathematical proof of convergence for the recursive attractor.
