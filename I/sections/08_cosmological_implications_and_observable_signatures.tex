\section{Cosmological Implications and Observable Signatures}
\label{sec:cosmological-implications}

\subsection{Gravitational Wave Interference Spectrum}

The recursive kernel \( K(\phi, \phi') \), derived from spin-sum amplitudes (Appendix~\ref{appendix:C}), predicts quantized coherence interference across cycles. This results in discrete dips in the stochastic gravitational wave (GW) spectrum at frequencies set by dominant SU(2) spin labels:
\[
f_j \sim \frac{\sqrt{j(j+1)}}{2\pi \ell_{\text{Pl}}}, \quad j \in \mathbb{Z}^+
\]
Each resonance corresponds to suppressed interference from dominant spin \( j \) in the boundary amplitude. The predicted suppression depth at each resonance is:
\[
\Delta \Omega_{\text{GW}}(f_j) \sim 10^{-12}
\]
These features differ from the smooth power-law predictions of inflationary models and may be observable by LISA, BBO, and DECIGO~\cite{maggiore_gravitational_2000, smith_gravitational_2006}.

\subsection{Non-Gaussianity in the CMB Bispectrum}

Recursive entanglement induces memory-dependent modulations in the CMB bispectrum:
\[
f_{\text{NL}}^{\text{rec}}(k) \sim \lambda_E |\langle \Psi_{n-1} | \Psi_n \rangle|^\alpha
\]
This yields:
\begin{itemize}
    \item scale-dependent \( f_{\text{NL}} \) with enhanced low-\( \ell \) amplitude,
    \item oscillatory structure tied to the entanglement-dependent kernel frequency \( \omega_0 \) (see Appendix~\ref{appendix:B}),
    \item potential correlation with existing low-\( \ell \) anomalies~\cite{planck2019inflation}.
\end{itemize}
Future constraints from CMB-S4 and the Simons Observatory will test this class of models~\cite{cmbs4forecast2019}.

\subsection{Parity-Violating Polarization in Void Environments}

In regions of partial memory collapse (e.g., large cosmic voids), coherence loss modulates the EB-mode polarization. The recursive model predicts:
\[
C_\ell^{EB} \propto \lambda_E^2 |\langle \Psi_{n-1} | \Psi_n \rangle|^2
\]
Observable effects include:
\begin{itemize}
    \item statistically significant EB-mode correlation in underdense regions,
    \item deviation from parity-conserving inflationary expectations,
    \item alignment of polarization axes with void boundaries.
\end{itemize}
These effects provide a clean falsification path, as standard inflation predicts vanishing EB correlation. The prediction can be tested via SKA and Euclid polarization–lensing cross-correlations~\cite{laureijs2011euclid, dewdney2009ska}.

\subsection{Late-Time Decoherence Drift}

If the memory kernel weakens over cosmic time, the model predicts residual structure in large-scale observables:
\begin{itemize}
    \item dark energy equation-of-state drift: \( \Delta w(z) \sim 0.01 \lambda_E \),
    \item void substructure suppression: \( N_{\text{sub}} \sim e^{-\lambda_E z} \),
    \item 1D power spectrum noise slope: \( P_{1D}(k) \propto k^{-0.4} \)
\end{itemize}
As coherence degrades, the influence of memory terms in the action diminishes, resulting in modified clustering and expansion history. These effects distinguish recursive coherence decay from scalar-field dark energy and can be constrained by DESI, LSST, and JWST~\cite{ivezic2019lsst, desi2016experiment}.

\subsection{Discriminators Against Competing Models}

We summarize observational differences:

\begin{table}[H]
\centering
\begin{tabular}{|l|c|c|c|}
\hline
\textbf{Observable} & \textbf{Recursive Model} & \textbf{Inflation} & \textbf{CCC} \\
\hline
GW Spectrum & Quantized suppression dips & Smooth spectrum & Suppressed, featureless \\
CMB \( f_{\text{NL}} \) & Memory-modulated, scale-dependent & Gaussian & Conformal re-scaling \\
Polarization (EB) & Void-aligned, parity-violating & Parity-symmetric & Absent due to conformal symmetry \\
Dark energy drift & Entropy-linked & Static or scalar-driven & Re-scaled entropy flow \\
\hline
\end{tabular}
\caption{Empirical discriminators between the recursive framework and competing models}
\end{table}

\subsection{Experimental Outlook}

Key tests include:
\begin{itemize}
    \item \textbf{LISA}: GW interference pattern detection near \( f \sim 10^{-3} \,\text{Hz} \),
    \item \textbf{CMB-S4}: Precise \( f_{\text{NL}}(k) \) measurements at low multipoles,
    \item \textbf{SKA and Euclid}: Void-induced EB correlation mapping,
    \item \textbf{LSST and JWST}: Void substructure and \( w(z) \) constraints.
\end{itemize}
These experiments collectively probe distinct facets of the recursive hypothesis, from coherence-filtered GW propagation to entropy-conditioned polarization signatures.

\subsection{Summary}

The recursive model predicts falsifiable deviations from both inflationary and conformal cyclic paradigms. Its observational structure arises from coherent memory transfer, entanglement constraints, and recursive boundary filtering. Detection or exclusion of these effects will determine the viability of the framework within the next decade.
