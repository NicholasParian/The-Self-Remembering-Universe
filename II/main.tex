\documentclass[11pt]{article}
\usepackage[margin=1in]{geometry}
\usepackage{amsmath, amssymb, amsthm}
\usepackage{authblk}
\usepackage{hyperref}
\usepackage{graphicx}
\usepackage{cite}
\usepackage{bm}
\usepackage{physics}
\usepackage{enumitem}
\usepackage{mathtools}
\usepackage{titlesec}

\titleformat{\section}
  {\normalfont\large\bfseries}{\thesection.}{1em}{}
\titleformat{\subsection}
  {\normalfont\normalsize\bfseries}{\thesubsection.}{1em}{}

\title{\textbf{II\\The Self-Remembering Universe\\Unified Dynamics in Recursive Spacetimes\footnote{This is a preprint draft of a scientific theory in development. All rights reserved by the author.}}}
\author[1]{Nicholas Parian}
\date{\today}

\begin{document}
\maketitle

\begin{abstract}
We develop a unified theoretical framework in which gravity, gauge interactions, scalar fields, and quantum measurement arise from a common principle: recursive coherence constrained by symmetry-preserving transitions across cosmological cycles. Extending the model introduced in \textit{The Self-Remembering Universe}, we formulate an action-based dynamics over a configuration space $\phi_n = (a_n, \varphi_n, \lambda_n, E_n, \gamma_n, \xi_n)$, where memory, entanglement, and observer resolution explicitly shape physical evolution. 

State transitions are governed by a non-Markovian kernel $K(\phi, \phi')$ filtered through a coherence-preserving subgroup $G_{\text{coh}} \subset \text{Rubik}(3)$, encoding which transformations are permitted within the recursive attractor basin. We construct a full recursive Lagrangian $\mathcal{A}_n[\phi]$ integrating Einstein–Hilbert, Yang–Mills, scalar, observer, and entropy penalty terms, all constrained by entanglement-based information flow and symmetry alignment.

Emergent forces appear as stable modes within the attractor structure, while measurement and decoherence are modeled as observer-projected entropy gradients. We derive observational predictions including gravitational wave echoes, CMB polarization alignment, and potential variation in coupling constants across cycles. This work establishes a pathway toward a recursive theory of everything and sets the stage for the generalization to biological and cognitive systems in Paper III.
\end{abstract}


\tableofcontents

\section{Introduction}

\subsection{Background and Motivation}

The preceding work, \textit{The Self-Remembering Universe}, introduced a recursive cosmological framework in which the universe evolves through discrete cycles, with each cycle preserving memory through entangled spacetime bridges. Central to that model was a non-Markovian transition kernel that filtered quantum decoherence through inherited attractor states, establishing a fixed-point configuration $\Psi^*(\phi)$ as the anchor of recursive coherence. The formalism unified loop quantum cosmology, Einstein–Rosen bridge thermodynamics, and observer-modulated decoherence into a coherent model of memory-driven cosmological dynamics.

The present paper extends that foundation toward a unified theory of fundamental interactions and symmetry-constrained evolution. Our goal is to incorporate gravity, gauge interactions, and quantum measurement into a single recursive action principle. We seek a formulation in which all physical forces—and the observer’s role in defining measurement outcomes—emerge from memory-preserving constraints on allowable transitions between configuration states.

To accomplish this, we introduce a generalized recursive configuration vector $\phi_n$ augmented with internal symmetry indices and observer resolution fields. We construct an extended action functional $\mathcal{A}_n[\phi]$ that includes gravitational, gauge, scalar, and informational terms, all subject to a recursive attractor constraint. A central insight of this work is that physical evolution is not governed solely by the equations of motion, but also by structural constraints on what transformations are \textit{permitted} by coherence logic.

\subsection{From Recursive Cosmology to Unification}

The central hypothesis of this work is that the set of allowed transitions $\phi_n \to \phi_{n+1}$ is filtered by an internal group structure akin to the Rubik cube group:
\[
G \subseteq \text{Rubik}(3) \cong ( \mathbb{Z}_3 \times \mathbb{Z}_3 )^5 \rtimes S_6,
\]
which models the finite, yet highly structured, space of coherent state evolutions. Within this space, only a subset $G_{\text{coh}}$ preserves alignment with the recursive attractor $\Psi^*(\phi)$. These symmetry constraints act as physical laws, projecting out incoherent configurations and enforcing a structured path through the recursive configuration space.

This leads to a novel formulation of unification: the fundamental interactions we observe—gravitational, electroweak, strong—are interpreted as \textit{modes of recursive alignment} within an attractor-constrained kernel. The observer, encoded explicitly through an entanglement tensor $O^{\mu\nu}$ and resolution index $\xi_n$, acts as both participant and filter in the recursive evolution.

We present here a complete recursive Lagrangian formalism, a group-theoretic filtering mechanism, and numerical strategies for simulating convergence to attractor states. We also derive observational predictions related to gravitational wave echoes, polarization alignment in the CMB, and entropy drift constraints, establishing pathways toward experimental falsifiability.

This paper serves as the second installment in a trilogy. Where Paper I formalized the cosmological memory kernel, Paper II defines the unification architecture. Paper III will extend this structure into biological, cognitive, and informational systems, proposing recursive coherence as a universal organizing principle across physical and emergent domains.


\section{Configuration Space and Symmetry Groups}

\subsection{Field Content}

We define an extended recursive configuration state vector
\[
\phi_n = \left( a_n, \varphi_n, \lambda_n, E_n, \gamma_n, \xi_n \right),
\]
where $(a_n, \varphi_n, \lambda_n, E_n)$ are inherited from the cosmological kernel formalism of Paper I, and $(\gamma_n, \xi_n)$ introduce internal symmetry parameters and observer-defined resolution indices, respectively. The internal degrees of freedom $\gamma_n$ represent generalized gauge orientations, while $\xi_n$ denotes the local interpretive resolution of the observer, which affects state decoherence and measurement thresholds~\cite{zurek2003decoherence, nielsen2010quantum}.

\subsection{Dimensional Embedding and Compactification}

We assume a 12-dimensional manifold $\mathcal{M}^{(12)}$, with compactification of higher-dimensional components analogous to string-theoretic frameworks~\cite{peskin1995introduction}. However, in our model, these compactified dimensions serve as recursive memory channels rather than simple geometric subspaces. The evolution of $\phi_n$ is constrained by both local dynamics and inherited memory traces via the transition kernel $K(\phi_{n}, \phi_{n-1})$~\cite{ashtekar2006quantum}.

\subsection{Gauge Symmetries and Interference Logic}

The recursive structure exhibits an underlying symmetry group $G$, which we conjecture to be a subgroup of the Rubik group:
\[
G \subseteq \text{Rubik}(3) \cong ( \mathbb{Z}_3 \times \mathbb{Z}_3 )^5 \rtimes S_6,
\]
where the non-abelian structure encodes both internal transformations and external permutation symmetries~\cite{joyner2008adventures}. This analogy is motivated by the finite, yet highly structured, configuration space of the 3x3x3 Rubik’s Cube, whose solvability depends not only on the current state but on the \textit{allowed transformation sequences}—mirroring the constraints on recursive state evolution.

In our model, the recursive transition operator $\mathcal{T}$ respects a coherence-preserving subgroup $G_{\text{coh}} \subset G$, such that only a subset of all theoretically possible transformations maintains alignment with the recursive attractor $\Psi^*(\phi)$. Observer input $\xi_n$ selects a resolution scale, effectively slicing the full configuration space into perceived macro- or microstates, akin to focusing on specific facelets or move sequences in the cube~\cite{zurek2003decoherence}.

This leads to the conjecture:

\begin{quote}
\textit{Recursive coherence imposes group-theoretic selection rules on allowed state transitions, equivalent to solving a Rubik-like constraint at each cycle.}
\end{quote}

The implication is that the universe does not explore all mathematically allowed paths, but rather filters them via a recursive interference logic grounded in symmetry constraints, memory alignment~\cite{maldacena2013cool}, and decoherence thresholds~\cite{zurek2003decoherence, nielsen2010quantum}.


\section{Unified Recursive Action}

\subsection{Extended Lagrangian Formalism}

We construct a recursive action functional
\[
\mathcal{A}_n[\phi] = \int_{\Sigma_n} \left( \mathcal{L}_{\text{EH}} + \mathcal{L}_{\text{YM}} + \mathcal{L}_{\text{scalar}} + \mathcal{L}_{\text{int}} + \mathcal{L}_{\text{obs}} \right) \, \mathrm{d}^4 x,
\]
defined on a compact 4-dimensional slice $\Sigma_n$ of the full 12-dimensional manifold $\mathcal{M}^{(12)}$. The terms represent:

- $\mathcal{L}_{\text{EH}}$: Einstein–Hilbert term for curvature dynamics~\cite{peskin1995introduction},
- $\mathcal{L}_{\text{YM}}$: generalized Yang–Mills term for gauge fields~\cite{peskin1995introduction},
- $\mathcal{L}_{\text{scalar}}$: scalar field contributions (e.g., inflaton, Higgs, axion),
- $\mathcal{L}_{\text{int}}$: interaction terms with coupling constants inherited via memory~\cite{ashtekar2006quantum},
- $\mathcal{L}_{\text{obs}}$: observer coupling term, which enforces decoherence constraints and recursive boundary conditions~\cite{zurek2003decoherence, nielsen2010quantum}.

Each Lagrangian component is subject to the transition rule
\[
\phi_{n+1} = \mathcal{T}[\phi_n] = \phi_n \oplus \delta \phi_n,
\]
where $\delta \phi_n$ is filtered by the coherence-preserving subgroup $G_{\text{coh}} \subset \text{Rubik}(3)$ defined in Section 2~\cite{joyner2008adventures}. This restricts allowed transformations in analogy to legal cube moves that preserve solvability under known constraints.

\subsection{Integration of Gravity and Gauge Fields}

We adopt the view that gauge fields and gravity arise from distinct symmetry sectors within the recursive configuration space. The total field strength tensor is decomposed as:
\[
\mathcal{F}_{\mu\nu} = R_{\mu\nu} + F_{\mu\nu}^{(a)} + D_{\mu\nu}^{(\gamma)},
\]
where $R_{\mu\nu}$ is the Ricci curvature, $F_{\mu\nu}^{(a)}$ denotes gauge field strength from nonabelian gauge group generators~\cite{peskin1995introduction}, and $D_{\mu\nu}^{(\gamma)}$ arises from internal Rubik-symmetry curvature due to recursive orientation shifts in the compactified $\gamma_n$ directions~\cite{joyner2008adventures}.

This additional term $D_{\mu\nu}^{(\gamma)}$ formally represents misalignment from the attractor $\Psi^*(\phi)$ and generates a geometric penalty term in the action. Thus, any recursive deviation that breaks coherence symmetry introduces an effective potential:
\[
V_{\text{misalign}}(\phi) = \lambda_C \cdot \mathrm{Tr}\left[ \left( \mathbb{I} - \mathcal{P}_{G_{\text{coh}}} \right) \delta \phi_n \right]^2,
\]
where $\mathcal{P}_{G_{\text{coh}}}$ is the projector onto the allowed symmetry subspace. The system is dynamically steered toward minimal misalignment—analogous to solving the cube by legal moves only.

\subsection{Observer Coupling and Entanglement Flow}

We incorporate the observer entanglement tensor $O_{\mu\nu}$ via the term
\[
\mathcal{L}_{\text{obs}} = \lambda_O \cdot \mathrm{Tr}\left( O^{\mu\nu} \nabla_\mu \phi \nabla_\nu \phi \right) - \xi_n \cdot S_{\text{ent}}(\rho_{\text{red}}),
\]
where $\lambda_O$ is a coupling parameter, $\xi_n$ is the recursive observer resolution (from Section 2), and $S_{\text{ent}}$ is the entanglement entropy across the Einstein–Rosen bridge from the previous cycle~\cite{maldacena2013cool, nielsen2010quantum}.

The key insight is that $\xi_n$ effectively selects a layer of reality to observe—coarse or fine—constraining allowable $\delta \phi_n$ much like a Rubik's Cube solver restricting move sequences to preserve partial solution states~\cite{zurek2003decoherence}. In this framework, the observer is not external, but embedded recursively, constraining the path of physical evolution in a thermodynamically consistent, symmetry-aware manner.





\section{Emergent Forces and Effective Potentials}

\subsection{Electroweak and Strong Interactions in Cycles}

Within the recursive framework, gauge interactions emerge from fiber bundle connections on internal symmetry spaces governed by the coherence-preserving group $G_{\text{coh}}$~\cite{peskin1995introduction}. Each allowed transformation $\delta \phi_n \in \mathfrak{g}_{\text{coh}}$ (the Lie algebra of $G_{\text{coh}}$) corresponds to a legitimate gauge rotation that preserves recursive solvability, analogous to legal sequences of Rubik cube moves that maintain face parity~\cite{joyner2008adventures}.

We define the gauge fields as recursive connections $A_\mu^{(a)}$ over the four-dimensional base manifold $\Sigma_n$, sourced by the projected memory current:
\[
J_\mu^{(a)} = \mathrm{Tr}\left[ \rho_{\text{red}} \, T^{(a)} \right],
\]
where $\rho_{\text{red}}$ is the reduced density matrix across the prior-cycle Einstein–Rosen bridge~\cite{maldacena2013cool}, and $T^{(a)}$ are the generators of the gauge group. 

The Yang–Mills term evolves via recursion-modulated coupling constants $g_n^{(a)}$, which depend dynamically on the cycle's fidelity $\lambda_n$, the entanglement eigenvalue $E_n$, and internal orientation $\gamma_n$:
\[
g_n^{(a)} = f^{(a)}(\lambda_n, E_n, \gamma_n),
\]
where $f^{(a)}$ is constrained by attractor alignment. This structure allows for variation in coupling strengths across cycles, potentially giving rise to observable cosmological evolution in fundamental constants~\cite{ashtekar2006quantum}.

\subsection{Gravity as Information Curvature}

Gravity in this framework emerges not solely from classical curvature, but from informational tension within the recursive field configuration. Specifically, we define an \emph{information-weighted curvature}:
\[
\widetilde{R}_{\mu\nu} := R_{\mu\nu} + \alpha \cdot \nabla_\mu \nabla_\nu \log \det \rho_{\text{red}},
\]
where $\alpha$ is a scaling parameter coupling entropy gradients to local geometry~\cite{verlinde2011emergent}. This correction term arises naturally from the observer–entanglement coupling in $\mathcal{L}_{\text{obs}}$, which penalizes divergence between geometric evolution and memory-constrained configuration states~\cite{zurek2003decoherence, nielsen2010quantum}.

Regions of high coherence flux (i.e., steep entanglement gradients) exhibit amplified curvature, encoding gravitational influence as a second-order response to recursive information flow. This mechanism supports a purely emergent interpretation of gravity from a quantum-informational substrate~\cite{verlinde2011emergent, maldacena2013cool}, in which no quantization of the gravitational field is assumed at the fundamental level.

\subsection{Dark Sector and Hidden Symmetries}

The recursive structure generically predicts a class of hidden degrees of freedom associated with misaligned transformations. Specifically, the difference between the full transformation group $G$ and the coherence-preserving subgroup $G_{\text{coh}}$,
\[
G \setminus G_{\text{coh}},
\]
encodes symmetry operations that do not preserve attractor alignment~\cite{joyner2008adventures}. These correspond to field modes $\chi^{(b)}$ that are thermodynamically suppressed or entropically decoupled from the observer-resolved physical domain~\cite{zurek2003decoherence}.

We define these dark sector excitations as:
\[
\chi^{(b)} \sim \delta \phi_n^{(b)} \quad \text{where} \quad \delta \phi_n^{(b)} \notin \mathfrak{g}_{\text{coh}}.
\]
Their contribution to the energy density is exponentially suppressed:
\[
\rho_{\chi} \propto \exp\left( - \lambda_C \cdot \| \mathcal{P}_{G_{\text{coh}}^\perp} \delta \phi_n \|^2 \right),
\]
unless forced into excitation by bounce dynamics, coherence collapse, or symmetry-breaking events near the recursive boundary. This formulation offers a natural candidate structure for dark matter and dark energy: coherent in structure but informationally inert relative to the memory-aligned physical frame.


\section{Entropy, Time, and Measurement}

\subsection{Thermodynamic Constraints}

The recursive action includes a constraint on entropy flux between cycles, enforced by an effective potential term:
\[
\mathcal{V}_{\text{ent}} = \lambda_S \cdot S_{\text{rel}}(\rho_{\phi_n} \, \| \, \rho_{\phi_{n-1}}),
\]
where \( S_{\text{rel}} \) is the quantum relative entropy between the current and prior configuration cycle:
\[
S_{\text{rel}}(\rho \| \sigma) = \Tr(\rho \log \rho - \rho \log \sigma)
\]
(see~\cite{nielsen2010quantum}).

This term ensures that informational evolution is coherent across bounces, enforcing a memory-preserving arrow of time. Entropy does not merely increase but is \textit{regulated recursively}, with informational penalties imposed on transitions that diverge from the attractor subspace of the previous cycle.

\subsection{Recursive Time and Emergent Causality}

In contrast to absolute or externally defined time, temporal structure in this model is emergent from the recursive update process itself. Each cycle index $n$ defines a discrete layer of historical memory, and the perceived forward flow of time corresponds to increasing overlap fidelity between recursive attractor states:
\[
\lambda_n := \left| \left\langle \Psi_n \middle| \Psi_{n-1} \right\rangle \right|^2.
\]
Temporal alignment is enforced by convergence toward the attractor, such that:
\[
\partial_n \phi_n \to 0 \quad \text{as} \quad \phi_n \to \Psi^*(\phi),
\]
resulting in an effective "freezing" of internal dynamics near the fixed point. This phase corresponds to the classical regime of causal stability. Conversely, deviations from attractor convergence increase decoherence rates and induce structural time dilation—mirroring cosmological expansion, collapse, or inflation-like transitions~\cite{ashtekar2006quantum}.

\subsection{Measurement and Decoherence in Unified Framework}

Measurement events are modeled as localized decoherence spikes:
\[
\mathcal{D}_x[\rho] = \rho - \rho^2 \approx \epsilon(x) \quad \text{when} \quad \xi_n(x) \gg 0,
\]
where $\xi_n(x)$ represents the observer’s resolution at spacetime location $x$~\cite{zurek2003decoherence}. High-resolution observation causes entropy gradient sharpening, effectively collapsing the local configuration space. This collapse is not absolute, but resolution-dependent:
\[
\rho \to \mathcal{P}_{\xi_n} \rho,
\]
with $\mathcal{P}_{\xi_n}$ denoting the projection operator governed by the observer's entropic threshold. Observation thus acts as a coherence filter, suppressing divergence from $\Psi^*(\phi)$ and modifying the recursive kernel itself.

This model is consistent with the principles of quantum Darwinism~\cite{zurek2003decoherence}, but is embedded within a cyclic, memory-retaining architecture. The observer does not merely extract classical outcomes from a quantum substrate, but recursively reshapes the configuration space from which future measurements emerge.







\section{Predictions and Experimental Pathways}

\subsection{Quantized Curvature Modes}

The recursive model predicts discrete curvature excitations arising from coherence-filtered memory evolution. These appear as quantized features in both the scalar and tensor perturbation spectra, particularly near cosmological bounce boundaries~\cite{ashtekar2006quantum}. The effective curvature spectrum is modulated by a coherence scale $j_0$, associated with spin foam amplitudes and entanglement fidelity:
\[
\Delta R \sim \frac{1}{\sqrt{j_0}} \cdot \mathcal{F}(a, \varphi, \lambda, E),
\]
where $\mathcal{F}$ is the Gaussian coherence filter encoded in the transition kernel $K(\phi, \phi')$.

Predicted observational signatures include:

\begin{itemize}[leftmargin=1.5em]
\item Low-$\ell$ CMB suppression due to recursive misalignment at early cycles~\cite{planck2018cmb}
\item Fine-structure anomalies in the angular power spectrum arising from attractor interference
\item Gravitational wave echo modulations with characteristic delay $\tau_E$ tied to $E_n$~\cite{cardoso2016echoes}
\end{itemize}

\subsection{Gravitational Wave and CMB Signatures}

In this framework, gravitational wave propagation is non-Markovian, inheriting temporal memory from previous cycles. The observed waveform $h_{\text{obs}}(t)$ becomes a convolution with a memory kernel $D(\tau, E)$:
\[
h_{\text{obs}}(t) = \int_0^t D(\tau, E) \cdot h(t - \tau) \, d\tau.
\]
This nonlocal structure can produce late-time gravitational echoes after black hole mergers, with predicted echo spacing set by $\tau_E$—a function of bridge entanglement~\cite{abedi2017echo, cardoso2016echoes}. These signals are experimentally accessible through post-merger analyses in LIGO–Virgo–KAGRA data.

In the CMB, recursive observer symmetry produces residual alignment in polarization axes and curvature modes. Specifically, we predict:

\begin{itemize}[leftmargin=1.5em]
\item Large-angle $E$-mode polarization vector alignment (e.g., quadrupole–octupole correlation)~\cite{planck2018cmb}
\item Phase-coherent suppression of tensor-to-scalar ratio $r$ at low multipoles $k$
\item Off-diagonal correlations in $TT$ and $TE$ power spectra reflecting residual memory alignment
\end{itemize}

These patterns arise from early-cycle entanglement constraints and coherence filtering—distinguishing the model from inflationary stochasticity.

\subsection{Testable Constraints on Extra Dimensions}

The 12-dimensional recursive configuration space implies evolving compactification scales $\ell_{\gamma}$ and $\ell_{\xi}$, tied to internal fields $(\gamma_n, \xi_n)$. Their effective scale in each cycle is governed by:
\[
\ell_{\gamma}^2(n) \sim \frac{1}{\lambda_C} \cdot \left\| \delta \phi_n^{(\gamma)} \right\|^2.
\]
Observable consequences may arise when misalignment from $G_{\text{coh}}$ induces coherent leakage from hidden dimensions into the visible sector.

Predicted effects include:

\begin{itemize}[leftmargin=1.5em]
\item Measurable deviations from Newtonian gravity at sub-millimeter scales, as in torsion-balance experiments~\cite{adelberger2009torsion}
\item Temporal drift in the fine-structure constant $\alpha$ due to recursive observer-state fluctuations~\cite{uzan2011varying}
\item Effective mass fluctuations in sterile neutrinos or axion-like particles via coherence modulation potentials~\cite{ringwald2012axions}
\end{itemize}

These predictions render the recursive model falsifiable and distinguish it from static higher-dimensional proposals that lack entropic evolution or observer-based projection logic.



\section{Discussion and Future Work}

\subsection{The Road to Paper III}

In this work, we have extended the recursive cosmological framework into a unified dynamical theory incorporating gravity, gauge fields, observer entanglement, and symmetry-constrained transformations within a 12-dimensional configuration space. A key innovation lies in the formal use of a Rubik-group-inspired symmetry constraint on legal recursive transitions~\cite{joyner2008adventures}, which governs both microscopic field evolution and macroscopic observables.

Paper III will focus on generalizing this framework to biological, informational, and emergent systems. The central question is whether recursive coherence, as a mechanism for stability and complexity, also underlies neural dynamics, ecological memory, and cognitive evolution~\cite{wolpert2001search, mitchell2021complexity}. Specifically, Paper III will address:

\begin{itemize}[leftmargin=1.5em]
\item Embedding recursive memory dynamics into neural and ecological systems
\item Mapping coherence attractors to evolutionary stable strategies and cognitive resilience
\item Generalizing the kernel formalism to cross-domain transitions (physical $\to$ informational)
\item Formulating a recursive selection principle that applies to observers, species, and systems alike
\end{itemize}

The long-term aim is to test whether memory-preserving recursive dynamics can serve as a unifying architecture not only across physical laws but across life and mind.

\subsection{Open Questions and Extensions}

Several foundational questions remain unresolved in the present model:

\begin{enumerate}[leftmargin=1.5em]
\item \textbf{Gauge–Attractor Coupling:} What determines the dynamical stability of $G_{\text{coh}}$ across recursive transitions, and can it evolve or fracture under entropic stress?
\item \textbf{Bounce Topology:} Are bounce geometries fixed by symmetry and entropy bounds, or probabilistically selected based on coherence phase?
\item \textbf{Entanglement Thermodynamics:} Can we derive a closed-form relationship between decoherence energy cost, fidelity degradation, and observer entropy load~\cite{zurek2003decoherence}?
\item \textbf{Experimental Bounds:} What is the empirical reach of gravitational wave interferometers (e.g., LISA, ET) and CMB polarization experiments (e.g., CMB-S4) in detecting recursive echo delays or early-cycle alignment~\cite{abedi2017echo, planck2018cmb}?
\end{enumerate}

These questions will shape the continued development of the recursive action and configuration space geometry and guide its intersection with phenomenological cosmology.

\bigskip

This paper establishes the groundwork for a full recursive Theory of Everything. By embedding coherence constraints into the symmetry algebra of reality, we propose a framework in which time, force, observation, and even measurement itself emerge from memory alignment across cycles. In the next installment, we will extend this architecture into the informational and biological domains to assess whether recursive coherence is a fundamental principle of complexity itself~\cite{mitchell2021complexity}.




\section*{Appendix A\\Recursive Symmetry and Group-Theoretic Constraints}
\addcontentsline{toc}{section}{Appendix A: Recursive Symmetry and Group-Theoretic Constraints}
\label{appendix:A}

\subsection*{A.1 Structure of the Rubik Group}

Let $\text{Rubik}(3)$ denote the full symmetry group of the 3×3×3 Rubik’s Cube. It is a finite, non-abelian group with
\[
|G| = 43,252,003,274,489,856,000,
\]
and is expressible as a semidirect product:
\[
\text{Rubik}(3) \cong ( \mathbb{Z}_3 \times \mathbb{Z}_3 )^5 \rtimes S_6,
\]
where the abelian part $(\mathbb{Z}_3 \times \mathbb{Z}_3)^5$ corresponds to facelet twist states, and the symmetric group $S_6$ encodes permutation symmetries among face centers~\cite{joyner2008adventures}. This structure provides a combinatorial testbed for modeling recursive symmetry logic: only transformations that preserve a solvability condition (i.e., continuity with an attractor path) are physically permitted within the recursive framework.

\subsection*{A.2 Recursive Symmetry Subgroup}

We define the coherence-preserving subgroup $G_{\text{coh}} \subseteq \text{Rubik}(3)$ as the set of transformations $\delta \phi_n$ that map $\phi_n$ into a submanifold aligned with the recursive attractor $\Psi^*(\phi)$:
\[
G_{\text{coh}} := \left\{ g \in \text{Rubik}(3) \; \middle| \; \| \mathcal{T}_g[\phi_n] - \Psi^*(\phi) \| < \epsilon \right\}.
\]
This defines a dynamic symmetry constraint: not all mathematically allowed transitions are physically meaningful. Recursive time evolution is filtered through $G_{\text{coh}}$, and coherence decay is induced when evolution deviates from this symmetry subspace.

\subsection*{A.3 Symmetry Filtering in the Transition Kernel}

Let $K(\phi, \phi')$ denote the transition kernel governing recursive evolution. Then:
\[
K(\phi, \phi') = \sum_{g \in G_{\text{coh}}} \mathcal{A}_g(\phi, \phi') \cdot e^{-S_{\text{rel}}(\phi, \phi')},
\]
where $\mathcal{A}_g$ represents the transition amplitude associated with transformation $g$, and $S_{\text{rel}}$ is the relative entropy between configurations. For transitions $g \notin G_{\text{coh}}$, we impose exponential suppression:
\[
\mathcal{A}_g \to 0 \quad \text{as} \quad g \notin G_{\text{coh}}.
\]
This ensures that physical evolution favors symmetry-preserving trajectories and avoids ergodic exploration of configuration space.

\subsection*{A.4 Projection Operators and Measurement Resolution}

Let $\mathcal{P}_{\xi}$ be the observer-resolution-dependent projection operator:
\[
\mathcal{P}_{\xi}[\rho] := \sum_{i \in \mathcal{B}_{\xi}} \ket{i}\bra{i} \rho \ket{i}\bra{i},
\]
where $\mathcal{B}_{\xi}$ denotes a measurement basis coarse-grained according to observer parameter $\xi$. This determines which degrees of freedom are visible under a given entropic threshold. Only transformations $g$ for which
\[
\mathcal{P}_{\xi} \circ \mathcal{T}_g[\phi]
\]
yields a valid eigenstate contribute to measurable outcomes.

Thus, both intrinsic symmetry filtering via $G_{\text{coh}}$ and extrinsic resolution via $\mathcal{P}_{\xi}$ jointly constrain physical state transitions. This is consistent with a Rubik-like logic of constrained maneuverability: many mathematically possible operations exist, but only those compatible with memory alignment and observational resolution manifest in the observable domain.




\section*{Appendix B\\Recursive Lagrangian Structure and Field Dynamics}
\addcontentsline{toc}{section}{Appendix B: Recursive Lagrangian Structure and Field Dynamics}
\label{appendix:B}

\subsection*{B.1 Total Action Decomposition}

The full recursive action over a cycle index \( n \) is defined as:
\[
\mathcal{A}_n[\phi] = \int_{\Sigma_n} \left( \mathcal{L}_{\text{EH}} + \mathcal{L}_{\text{YM}} + \mathcal{L}_{\text{scalar}} + \mathcal{L}_{\text{int}} + \mathcal{L}_{\text{obs}} + \mathcal{L}_{\text{ent}} \right) \, \mathrm{d}^4 x,
\]
where each Lagrangian density term corresponds to a structural role in the recursive dynamics~\cite{peskin1995introduction, nielsen2010quantum}:

\begin{itemize}[leftmargin=1.5em]
\item $\mathcal{L}_{\text{EH}}$: Einstein–Hilbert term encoding classical spacetime curvature:
\[
\mathcal{L}_{\text{EH}} = \frac{1}{2\kappa} R,
\]
where \( R \) is the Ricci scalar and \( \kappa = 8\pi G \).

\item $\mathcal{L}_{\text{YM}}$: Gauge field dynamics with memory-modulated coupling:
\[
\mathcal{L}_{\text{YM}} = -\frac{1}{4} \sum_a \left( g_n^{(a)} \right)^{-2} F_{\mu\nu}^{(a)} F^{\mu\nu}_{(a)},
\]
with $g_n^{(a)} = f^{(a)}(\lambda_n, E_n, \gamma_n)$ representing coherence-adaptive coupling constants~\cite{ashtekar2006quantum}.

\item $\mathcal{L}_{\text{scalar}}$: Scalar field evolution including recursive inheritance:
\[
\mathcal{L}_{\text{scalar}} = \frac{1}{2} g^{\mu\nu} \partial_\mu \varphi \, \partial_\nu \varphi - V(\varphi, \lambda_n),
\]
where \( V(\varphi) \) is a potential function modulated by memory alignment with $\Psi^*(\phi)$.

\item $\mathcal{L}_{\text{int}}$: Cross-field interactions, including coupling to internal symmetry coordinates:
\[
\mathcal{L}_{\text{int}} = - \sum_{a,b} \beta_{ab} \cdot \varphi \, \bar{\psi}_a \psi_b + \sum_{i} \xi_n \cdot \varphi \, \partial_\mu \gamma^i,
\]
where \( \gamma^i \) are compactified internal degrees of freedom and $\beta_{ab}$ are inherited interaction coefficients.

\item $\mathcal{L}_{\text{obs}}$: Observer entanglement and decoherence regulation:
\[
\mathcal{L}_{\text{obs}} = \lambda_O \cdot O^{\mu\nu} \nabla_\mu \phi \nabla_\nu \phi - \xi_n \cdot S_{\text{ent}}(\rho_{\text{red}}),
\]
where $O^{\mu\nu}$ encodes entanglement geometry and $\xi_n$ modulates local resolution thresholds~\cite{zurek2003decoherence}.

\item $\mathcal{L}_{\text{ent}}$: Recursive entropy penalty enforcing memory alignment:
\[
\mathcal{L}_{\text{ent}} = - \lambda_S \cdot S_{\text{rel}}(\rho_{\phi_n} \| \rho_{\phi_{n-1}}),
\]
where $S_{\text{rel}}$ is the quantum relative entropy~\cite{nielsen2010quantum}, ensuring consistency across cycles.
\end{itemize}

\subsection*{B.2 Recursive Evolution Equation}

The configuration field $\phi_n$ evolves recursively:
\[
\phi_{n+1} = \phi_n \oplus \delta \phi_n,
\]
where $\delta \phi_n$ is filtered by a variational principle enforcing coherence symmetry:
\[
\delta \mathcal{A}_n[\phi] = 0 \quad \text{subject to} \quad \delta \phi_n \in \mathfrak{g}_{\text{coh}}.
\]
This ensures that only allowed transformations contribute to the Euler–Lagrange dynamics:
\[
\frac{\delta \mathcal{A}_n}{\delta \phi} = 0,
\]
and that recursive evolution remains confined to the attractor-preserving symmetry subspace (see Appendix A).

\subsection*{B.3 Effective Potential Landscape and Stability}

The recursive attractor $\Psi^*(\phi)$ defines a fixed point in configuration space, toward which recursive evolution converges. Local stability is determined by the Hessian of the action:
\[
\mathcal{H}_{ij} = \frac{\partial^2 \mathcal{A}_n}{\partial \phi^i \partial \phi^j} \Bigg|_{\phi = \Psi^*(\phi)}.
\]
Lyapunov stability is satisfied if $\mathcal{H}$ is positive semi-definite in the subspace defined by the coherence projection:
\[
\phi \in \mathcal{P}_{G_{\text{coh}}}(\mathcal{C}_n).
\]
This structure ensures that evolution favors recurrence and memory preservation rather than chaotic divergence, consistent with observed large-scale coherence in physical law.


\section*{Appendix C\\Recursive Kernel Implementation and Attractor Derivation}
\addcontentsline{toc}{section}{Appendix C: Recursive Kernel Implementation and Attractor Derivation}
\label{appendix:C}

\subsection*{C.1 Kernel Structure and Configuration Space}

We define the extended recursive configuration vector as:
\[
\phi_n = (a_n, \varphi_n, \lambda_n, E_n, \gamma_n, \xi_n),
\]
where:

\begin{itemize}[leftmargin=1.5em]
  \item $a_n$: Discrete scale factor (from LQC or geometric phase space)~\cite{ashtekar2006quantum}
  \item $\varphi_n$: Scalar field value(s)
  \item $\lambda_n$: Overlap fidelity with previous attractor state
  \item $E_n$: Entanglement eigenvalue across the Einstein–Rosen bridge~\cite{maldacena2013cool}
  \item $\gamma_n$: Internal symmetry orientation (e.g., compactified gauge index)
  \item $\xi_n$: Observer resolution scale affecting decoherence~\cite{zurek2003decoherence}
\end{itemize}

\subsection*{C.2 Transition Kernel Definition}

The recursive transition kernel governing configuration evolution is defined by:
\[
K(\phi, \phi') = \sum_{g \in G_{\text{coh}}} \mathcal{A}_g(\phi, \phi') \cdot \exp\left[ -S_{\text{rel}}(\rho_\phi \| \rho_{\phi'}) \right],
\]
where:

\begin{itemize}[leftmargin=1.5em]
  \item $g \in G_{\text{coh}}$: Legal transformation under symmetry constraints (see Appendix A)
  \item $\mathcal{A}_g(\phi, \phi')$: Amplitude for transformation $g$
  \item $S_{\text{rel}}$: Quantum relative entropy~\cite{nielsen2010quantum}
\end{itemize}

The kernel encodes both a quantum path integral propagator and a coherence-preserving memory filter.

\subsection*{C.3 Numerical Simulation Strategy}

The evolution is simulated iteratively across cycle index $n$:

1. **Initialization**:
   \[
   \phi_0 \gets (\bar{a}, \bar{\varphi}, \lambda_0 = 1, E_0, \gamma_0, \xi_0),
   \]
   where initial conditions $\bar{a}$, $\bar{\varphi}$ are drawn from LQC or observational priors.

2. **Recursive Update**:
   \[
   \phi_{n+1} = \phi_n \oplus \delta \phi_n, \quad \text{with} \quad \delta \phi_n \sim \mathcal{F}(\phi_n, \Psi^*(\phi)),
   \]
   filtered through $K(\phi, \phi')$ and projected onto $G_{\text{coh}}$.

3. **Fidelity Tracking**:
   \[
   \lambda_n := \left| \left\langle \Psi_n \middle| \Psi_{n-1} \right\rangle \right|^2,
   \]
   monitoring convergence toward the attractor $\Psi^*(\phi)$.

4. **Entropy Monitoring**:
   \[
   S_n = -\Tr\left( \rho_{\text{red}}^{(n)} \log \rho_{\text{red}}^{(n)} \right), \quad
   E_n := \sqrt{S_n},
   \]
   with entropy used to compute entanglement energy and decoherence rate.

\subsection*{C.4 Attractor Derivation via Variational Convergence}

The attractor state is defined variationally as:
\[
\Psi^*(\phi) = \arg \min_{\phi \in \mathcal{P}_{G_{\text{coh}}}(\mathcal{C}_n)} \mathcal{A}_n[\phi],
\]
and satisfies the recursive fixed-point condition:
\[
\phi_{n+1} = \phi_n = \Psi^*(\phi).
\]

To numerically converge toward the attractor, we use filtered gradient descent:
\[
\delta \phi_n = -\eta \cdot \nabla_{\phi} \mathcal{A}_n[\phi_n] + \mathcal{P}_{G_{\text{coh}}^\perp} \left[ \delta_{\text{noise}} \right],
\]
where $\eta$ is a learning rate and $\delta_{\text{noise}}$ represents entropy-weighted stochastic variation suppressed outside $G_{\text{coh}}$.

\subsection*{C.5 Entropy Compensation Constraint}

To maintain conservation across Einstein–Rosen bridge transitions, we impose a thermodynamic consistency condition:
\[
\frac{d}{dn} S_{\text{total}}^{(n)} + \frac{\delta Q_{\text{Hawking}}^{(n)}}{T_n} = 0,
\]
interpreted as entropy compensation via Hawking-like radiation at recursive bounce boundaries~\cite{verlinde2011emergent}. This constraint enforces entropy balance and prevents unbounded memory accumulation.

\subsection*{C.6 Convergence Criterion}

We define convergence toward the attractor as:
\[
\lambda_n \to 1, \quad \Delta S_n < \epsilon, \quad \|\phi_n - \phi_{n-1}\| < \delta,
\]
ensuring high-fidelity memory alignment, entropy stabilization, and recursive coherence. These thresholds are used to halt or rescale evolution in numerical simulations and inform stability diagnostics across the configuration manifold.



\section*{Appendix D\\Notation and Operator Glossary}
\addcontentsline{toc}{section}{Appendix D: Notation and Operator Glossary}
\label{appendix:D}

\subsection*{D.1 Configuration Variables}

\begin{itemize}[leftmargin=1.5em]
  \item $a_n$ — Discrete scale factor at cycle index $n$ (from LQC~\cite{ashtekar2006quantum})
  \item $\varphi_n$ — Scalar field value (e.g., inflaton, Higgs)
  \item $\lambda_n$ — Overlap fidelity between consecutive attractor states:
  \[
  \lambda_n := \left| \left\langle \Psi_n \middle| \Psi_{n-1} \right\rangle \right|^2
  \]
  \item $E_n$ — Entanglement eigenvalue across Einstein–Rosen bridge:
  \[
  E_n := \sqrt{S(\rho_{\text{red}})} = \sqrt{ -\Tr(\rho_{\text{red}} \log \rho_{\text{red}}) }
  \]
  ~\cite{maldacena2013cool}
  \item $\gamma_n$ — Internal symmetry index (e.g., gauge phase or compactified dimension)
  \item $\xi_n$ — Observer resolution parameter (affects decoherence~\cite{zurek2003decoherence})
  \item $\phi_n$ — Full recursive configuration vector:
  \[
  \phi_n = (a_n, \varphi_n, \lambda_n, E_n, \gamma_n, \xi_n)
  \]
\end{itemize}

\subsection*{D.2 Operators and Tensors}

\begin{itemize}[leftmargin=1.5em]
  \item $\mathcal{T}$ — Recursive transition operator: $\phi_{n+1} = \mathcal{T}[\phi_n]$
  \item $\mathcal{P}_{G_{\text{coh}}}$ — Projector onto the coherence-preserving symmetry subspace
  \item $\mathcal{P}_{\xi}$ — Projection operator at observer resolution $\xi$
  \item $O^{\mu\nu}$ — Observer entanglement tensor (couples to decoherence terms)
  \item $R_{\mu\nu}$ — Ricci curvature tensor
  \item $\widetilde{R}_{\mu\nu}$ — Information-weighted curvature:
  \[
  \widetilde{R}_{\mu\nu} := R_{\mu\nu} + \alpha \cdot \nabla_\mu \nabla_\nu \log \det \rho_{\text{red}}
  \]
  ~\cite{verlinde2011emergent}
  \item $F_{\mu\nu}^{(a)}$ — Field strength tensor for gauge field $a$
  \item $\rho$ — Quantum density matrix (total system)
  \item $\rho_{\text{red}}$ — Reduced density matrix across ER bridge
  \item $S(\rho)$ — Von Neumann entropy
  \item $S_{\text{rel}}(\rho \| \sigma)$ — Quantum relative entropy~\cite{nielsen2010quantum}
  \item $K(\phi, \phi')$ — Recursive transition kernel (see Section 4)
  \item $\Psi^*(\phi)$ — Attractor state of recursive evolution
\end{itemize}

\subsection*{D.3 Groups and Symmetry Spaces}

\begin{itemize}[leftmargin=1.5em]
  \item $\text{Rubik}(3)$ — Full Rubik's Cube group:
  \[
  \text{Rubik}(3) \cong (\mathbb{Z}_3 \times \mathbb{Z}_3)^5 \rtimes S_6
  \]
  (see~\cite{joyner2008adventures})
  \item $G_{\text{coh}}$ — Coherence-preserving symmetry subgroup
  \item $\mathfrak{g}_{\text{coh}}$ — Lie algebra of $G_{\text{coh}}$
\end{itemize}


\subsection*{D.4 Lagrangian Terms}

\begin{itemize}[leftmargin=1.5em]
  \item $\mathcal{L}_{\text{EH}}$ — Einstein–Hilbert term: $\frac{1}{2\kappa} R$
  \item $\mathcal{L}_{\text{YM}}$ — Yang–Mills term for gauge dynamics
  \item $\mathcal{L}_{\text{scalar}}$ — Scalar field kinetic and potential terms
  \item $\mathcal{L}_{\text{int}}$ — Field interaction and coupling terms
  \item $\mathcal{L}_{\text{obs}}$ — Observer coupling and decoherence regulation
  \item $\mathcal{L}_{\text{ent}}$ — Recursive entropy penalty term
  \item $\mathcal{A}_n[\phi]$ — Total recursive action at index $n$
\end{itemize}

\subsection*{D.5 Miscellaneous Notation}

\begin{itemize}[leftmargin=1.5em]
  \item $\oplus$ — Recursive blending operator (coherence-weighted XOR)
  \item $\eta$ — Learning rate for variational descent
  \item $\delta_{\text{noise}}$ — Thermodynamic noise vector orthogonal to $G_{\text{coh}}$
  \item $\ell_\gamma$ — Effective compactification scale for $\gamma$
  \item $\tau_E$ — Gravitational echo delay, related to $E_n$
\end{itemize}


\section*{Disclosure on the Use of AI}
\addcontentsline{toc}{section}{Disclosure on the Use of AI}
\label{appendix:disclosure}

Portions of this manuscript—including the development, refinement, formatting of mathematical expressions, narrative structure, and citation management—were produced in collaboration with OpenAI's GPT-4 model (ChatGPT), Google's Gemini 2.0, and DeepSeek R1. The human author, Nicholas Parian, guided the conceptual framework, directed the line of inquiry, posed the core hypotheses, and curated the final scientific content.

The use of artificial intelligence was instrumental in accelerating the writing, organizing technical arguments, and cross-referencing related literature. However, all original theoretical contributions, interpretations, and decisions regarding inclusion, emphasis, and framing were made by the human author.

This disclosure is provided in the interest of transparency and to acknowledge the evolving role of large language models in academic research and writing. The author assumes full responsibility for the accuracy, novelty, and scientific validity of the material presented.


\section*{References}
\addcontentsline{toc}{section}{References}
\bibliographystyle{unsrt}
\bibliography{references}

\end{document}
