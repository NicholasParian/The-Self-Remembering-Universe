\section{Emergent Forces and Effective Potentials}
\label{sec:emergent-forces}

\subsection{Gauge Interactions from Toroidal Phase Alignment}

In the recursive framework, gauge interactions arise from the preservation of phase coherence along toroidal trajectories in configuration space. Allowed field transformations correspond to phase-aligned loops in the compactified directions \( \gamma_n \), embedded within the toroidal submanifold \( \mathcal{T}_{\text{coh}} \subset \mathcal{C}_n \). These gauge-compatible deformations are filtered by the attractor condition \( \Psi^*(\phi) \), ensuring that only coherence-preserving transitions contribute dynamically.

We define the emergent gauge field \( A_\mu^{(a)} \) as a connection over the 4D base manifold \( \Sigma_n \), sourced by a memory-projected current:
\[
J_\mu^{(a)} = \mathrm{Tr}\left[ \rho_{\text{red}} \, T^{(a)} \right],
\]
where:
- \( \rho_{\text{red}} \) is the reduced density matrix across the ER bridge from the prior cycle,
- \( T^{(a)} \) are the generators of the internal gauge algebra,
- \( J_\mu^{(a)} \) encodes the effective memory flow across symmetry modes.

Gauge couplings become memory-dependent quantities:
\[
g_n^{(a)} = f^{(a)}(\lambda_n, E_n, \gamma_n),
\]
where \( f^{(a)} \) is a coherence-modulated function shaped by cycle fidelity \( \lambda_n \), entanglement eigenvalue \( E_n \), and internal orientation \( \gamma_n \). Variations in these parameters across cycles lead to controlled evolution of interaction strengths, constrained by attractor stability.

\subsection{Gravity as Curvature Response to Information Tension}

Gravitational influence in this model is interpreted as a second-order response to coherence gradients within the recursive configuration field. We define an information-weighted Ricci tensor:
\[
\widetilde{R}_{\mu\nu} := R_{\mu\nu} + \alpha \cdot \nabla_\mu \nabla_\nu \log \det \rho_{\text{red}},
\]
where \( \alpha \) is a dimensionless coupling that links entropy curvature to spacetime geometry.

This correction term penalizes geometric evolution that diverges from attractor-aligned information flow. In regions where memory fails to propagate cleanly—such as near bounce events or coherence collapse—the modified curvature term introduces an effective backreaction, encoding the gravitational imprint of recursive information breakdown.

This formalism supports a thermodynamically emergent view of gravity, where curvature acts as the stress field of recursive memory deformation.

\subsection{Coherence Structure as the Generator of Physical Law}

Both gauge and gravitational interactions are interpreted here as structured consequences of recursive coherence filtering. Physical forces emerge not from fundamental symmetry groups imposed a priori, but from phase alignment requirements that permit stable attractor evolution.

The recursive transition kernel \( K(\phi, \phi') \) acts as a filter over configuration space, enforcing coherence compatibility:
\[
K(\phi, \phi') \sim \exp\left( -\frac{\Delta \theta^2}{2\sigma_\theta^2} - \frac{S_{\text{rel}}(\rho_\phi \| \rho_{\phi'})}{\lambda_S} \right),
\]
where \( \Delta \theta \) is the phase deviation between configurations and \( \lambda_S \) governs entropy sensitivity.

Thus, gauge fields correspond to configurations with permissible winding numbers on the coherence torus, while gravity reflects the system’s response to memory misalignment. No additional dark sector content is introduced in this section, as such modes are separately treated in Section~\ref{sec:dark-sector}.
