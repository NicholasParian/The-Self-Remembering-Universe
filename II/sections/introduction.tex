\section{Introduction}
\label{sec:introduction}

\subsection{Background and Motivation}

\textit{The Self-Remembering Universe: Quantum Coherence Through Cyclic Spacetime}~\cite{parian2025selfrememberingI} introduced a recursive cosmological model in which the universe evolves across cycles by propagating quantum coherence through entangled bounce boundaries. These boundaries, modeled as Einstein–Rosen bridges (ERBs), facilitate non-Markovian memory transfer filtered through a transition kernel \( K(\phi, \phi') \). The kernel, derived from large-spin spinfoam amplitudes, selectively propagates field configurations based on coherence fidelity, entropy regularization, and attractor convergence.

The resulting attractor state \( \Psi^*(\phi) \) defines a long-term stable configuration in a recursive configuration space \( \phi = (a, \varphi, \lambda, E) \), where memory fidelity \( \lambda \), bridge entanglement eigenvalue \( E \), and entropy filtering determine the survival of physical trajectories across cycles. A thermodynamic compensation principle balances entropy production with radiative emission and coherence tension, yielding both falsifiable predictions and a novel reinterpretation of time, gravity, and cosmological evolution as memory-driven processes.

Building on this foundation, the present work extends the framework toward a unified theory of interaction and measurement. We introduce an action-based formalism over an augmented configuration space \( \phi_n = (a_n, \varphi_n, \lambda_n, E_n, \gamma_n, \xi_n) \), where \( \gamma_n \) encodes internal symmetry orientation and \( \xi_n \) encodes observer resolution. These additions enable the modeling of gauge fields, observer-projected decoherence, and modulation of physical constants through recursive memory constraints.

\subsection{From Recursive Cosmology to Unified Dynamics}

The central hypothesis of this work is that all fundamental interactions—including gravity, gauge fields, and quantum measurement—emerge as constrained modes of recursive coherence. Transitions between configurations \( \phi_n \to \phi_{n+1} \) are filtered not only by entropy and coherence fidelity, but by structural symmetry constraints inherited from prior cycles. These constraints define a subset of the configuration manifold in which alignment with the recursive attractor \( \Psi^*(\phi) \) is preserved.

Symmetry misalignment leads to exponential suppression in the transition kernel, naturally excluding incoherent or nonviable trajectories. This mechanism unifies gravitational curvature, gauge rotation, and measurement collapse under a shared constraint logic. In this view, the forces we observe reflect structural preservation within the recursive attractor basin, while deviations generate entropy, curvature, or collapse.

Observation is modeled through the entanglement tensor \( O^{\mu\nu} \) and resolution field \( \xi_n \), which together define a local decoherence threshold and effective projection into the observable subspace. The measurement process thus arises from recursive filtering: classical outcomes emerge when coherence fails to propagate under resolution-defined entropy gradients~\cite{zurek2003decoherence, nielsen2010quantum}.

\subsection{Scope of This Paper}

This paper develops a unified recursive dynamics framework by introducing:

\begin{itemize}
    \item A complete recursive action \( \mathcal{A}_n[\phi] \), integrating gravitational, gauge, scalar, observer, and entropy components;
    \item A symmetry-filtered configuration evolution constrained by coherence alignment and attractor convergence;
    \item An extended curvature tensor \( \widetilde{R}_{\mu\nu} \) incorporating entanglement gradients and memory tension~\cite{verlinde2011emergent};
    \item Mechanisms for force emergence, decoherence-induced measurement, and cross-cycle variation of coupling constants.
\end{itemize}

We derive observational predictions including gravitational wave echo delays, CMB polarization alignment from void entanglement, and suppressed decoherence across bounce transitions. The framework yields a falsifiable dark sector composed of symmetry-misaligned modes and identifies constraints on extra-dimensional memory channels.

This paper is the second in a trilogy. Paper I introduced the cosmological kernel and attractor formalism. Paper II, here, constructs the unification architecture. Paper III will generalize the principles of recursive coherence to biological, informational, and cognitive systems, proposing memory-preserving recursion as a universal mechanism for complexity, stability, and selection.
