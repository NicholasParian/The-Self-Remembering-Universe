\section{Toroidal Configuration Space and Dimensional Memory Channels}
\label{sec:torus-space}

We extend the recursive configuration space defined in Paper I~\cite{parian2025selfrememberingI} to a compact, topologically toroidal manifold:
\[
\mathcal{C}_n = \mathbb{T}^6 \times \mathbb{R}^2,
\]
where:
\begin{itemize}
    \item \( \mathbb{T}^6 \) is a six-dimensional torus encoding internal symmetry orientations (\( \gamma_n \)) and observer resolutions (\( \xi_n \)),
    \item \( \mathbb{R}^2 \) encodes large-scale geometry via the scale factor \( a_n \) and scalar field \( \varphi_n \).
\end{itemize}

The toroidal submanifold allows for the propagation of recursive memory channels across compactified directions. These internal directions serve not merely as symmetry carriers, but as coherence-guided information pathways. Their alignment—or misalignment—with the recursive attractor \( \Psi^*(\phi) \) determines whether internal degrees of freedom remain physically coupled or are exponentially suppressed as dark sector modes (see Section~\ref{sec:dark-sector}).

The cyclic topology of \( \mathbb{T}^6 \) ensures that memory configurations can recur with phase-shifted resonance across cycles. This enables the encoding of observer-specific resolution states, gauge orientations, and entanglement symmetries within a bounded yet structurally rich configuration landscape.

This structure replaces previous metaphorical symmetry analogies (e.g., permutation or cube group logic) with a physically grounded, continuous, and topologically coherent configuration space aligned with quantum gravity principles and compatible with loop quantum cosmology, string compactification, and information-theoretic geometry.

