\section{Dark Sector as Coherence Misalignment in Toroidal Configuration Space}
\label{sec:dark-sector}

The recursive attractor structure developed in this framework defines a toroidal submanifold \( \mathcal{T}_{\text{coh}} \subset \mathcal{C}_n \), representing configurations that preserve phase alignment and memory fidelity across cycles. However, the full recursive configuration space \( \mathcal{C}_n = (a_n, \varphi_n, \lambda_n, E_n, \gamma_n, \xi_n) \) admits trajectories that fall outside this coherence-preserving basin.

These misaligned trajectories correspond to \textit{dark sector modes}: configurations that are not decohered into the observer-resolved subspace but persist as \textit{off-attractor echoes} in the higher-dimensional memory topology.

\subsection*{Coherence-Filtered Suppression Mechanism}

Dark sector excitations arise when configurations fail to satisfy the phase coherence and entropy compensation criteria required for recursive survival. These modes are filtered out by the coherence kernel \( \mathcal{F}(\phi, \phi') \), which penalizes deviations from the attractor subspace:
\[
\mathcal{F}(\phi, \phi') = \exp\left( -\frac{\Delta \theta^2}{2\sigma_\theta^2} - \frac{(E - E')^2}{2\sigma_E^2} \right),
\]
where \( \Delta \theta \) denotes phase misalignment across the toroidal coherence cycle. Configurations with large \( \Delta \theta \) or entropy discontinuity are exponentially suppressed in the transition kernel:
\[
K(\phi, \phi') \sim \mathcal{F}(\phi, \phi') \cdot e^{-S_{\text{rel}}(\rho_\phi \| \rho_{\phi'})}.
\]

\subsection*{Activation During Bounce and Collapse}

Under normal conditions, dark sector modes are thermodynamically suppressed. However, near bounce events, coherence rupture, or dimensional string fracture (see Appendix~\ref{appendix:C}), the attractor basin can destabilize. This induces:

\begin{itemize}
  \item Temporary activation of off-attractor configurations
  \item Leakage of misaligned modes from compactified sectors
  \item Radiative imprint from entropy-overloaded boundary conditions
\end{itemize}

Such transitions may produce observable phenomena (e.g., gravitational bursts, curvature shocks, or memory scars) in the visible sector as a result of latent coherence ejection.

\subsection*{Geometric Interpretation and Energy Density}

In the toroidal embedding of configuration space, coherence-preserving trajectories wind smoothly around a compact manifold. Dark sector modes correspond to discontinuous excursions or destructive interference nodes, occupying quasi-inaccessible regions outside the coherence band. The effective energy density of such modes is governed by:
\[
\rho_{\chi} \propto \exp\left( -\lambda_C \cdot \Delta \theta^2 \right),
\]
where \( \lambda_C \) is the coherence tension and \( \Delta \theta \) is the phase offset from the attractor loop.

\subsection*{Phenomenological Implications}

This formulation predicts that the dark sector:

\begin{itemize}
  \item Does not require new fields or particles beyond the configuration space variables \( \phi \)
  \item Becomes active only under coherence breakdown or tension-overload collapse
  \item May leave subtle observational signatures, including gravitational memory echoes, entropy drift, and anomalous void alignments
  \item Is naturally embedded in the recursive memory dynamics of the cosmological system
\end{itemize}

Rather than positing dark matter and dark energy as exotic entities, this framework interprets them as the shadow of coherence failure: \textit{non-interfering branches} of the universe’s own memory.
