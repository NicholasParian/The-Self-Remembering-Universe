\section{Entropy, Time, and Measurement}

\subsection{Thermodynamic Constraints on Recursive Evolution}

The recursive action imposes an entropy compensation constraint between successive cycles. This is enforced via a potential term:
\[
\mathcal{V}_{\text{ent}} = \lambda_S \cdot S_{\text{rel}}(\rho_{\phi_n} \| \rho_{\phi_{n-1}}),
\]
where $S_{\text{rel}}$ is the quantum relative entropy:
\[
S_{\text{rel}}(\rho \| \sigma) = \Tr(\rho \log \rho - \rho \log \sigma),
\]
and $\lambda_S$ is a Lagrange multiplier enforcing bounded memory drift.

This constraint ensures that entropy increase is not unconstrained but balanced against coherence fidelity. Excess divergence from prior states incurs an action penalty, enforcing the thermodynamic principle:
\[
\Delta S_{\text{fwd}} = \Delta S_{\text{mem}}.
\]

\subsection{Recursive Time and Attractor Convergence}

Time in this framework is emergent and discrete, indexed by the cycle number $n$. The arrow of time corresponds to the forward propagation of coherence fidelity:
\[
\lambda_n := \left| \left\langle \Psi_n \middle| \Psi_{n-1} \right\rangle \right|^2.
\]
Temporal alignment is driven by convergence toward the recursive attractor $\Psi^*(\phi)$:
\[
\partial_n \phi_n \to 0 \quad \text{as} \quad \phi_n \to \Psi^*(\phi).
\]
This limit corresponds to entropy saturation and maximal recursive coherence. Deviation from the attractor induces decoherence and structural time dilation, interpreted geometrically as expansion or curvature acceleration.

\subsection{Measurement as Resolution-Dependent Collapse}

Measurement arises from resolution-dependent decoherence triggered by the observer field $\xi_n$. The localized decoherence map is:
\[
\mathcal{D}_x[\rho] = \rho - \rho^2 \approx \epsilon(x) \quad \text{when} \quad \xi_n(x) \gg 0.
\]
High-resolution observers induce sharper entropy gradients, causing selective projection:
\[
\rho \to \mathcal{P}_{\xi_n} \rho,
\]
where $\mathcal{P}_{\xi_n}$ is a coarse-grained projection operator determined by the observer's resolution scale.

This process is consistent with quantum Darwinism~\cite{zurek2003decoherence}, but extended into a cyclic recursive setting. Observation is not a terminal act but part of a feedback loop that reshapes the recursive kernel and memory trajectory.

\subsection{Decoherence as Kernel Modification}

Decoherence modifies the recursive transition kernel $K(\phi, \phi')$ by narrowing the Gaussian envelope in the resolution-weighted direction. Observers with high $\xi_n$ restrict the support of $K$, enforcing local collapse and suppressing transitions that diverge from their resolved subspace.

This leads to an adaptive kernel structure:
\[
K(\phi, \phi') \to K_{\xi_n}(\phi, \phi') = K(\phi, \phi') \cdot \mathbb{1}_{\text{aligned with } \mathcal{P}_{\xi_n}},
\]
where only transitions consistent with observer resolution and memory coherence contribute to evolution.

\subsection{Summary}

Recursive time is a function of fidelity evolution. Entropy constraints ensure coherence preservation, while observation projects future states into resolution-dependent subspaces. Decoherence is not external collapse, but internal modulation of the transition structure governing recursive state propagation.
